%
% Trabalhos relacionados
%
\chapter{Trabalhos relacionados}

Pode-se observar que a área que possui uma grande quantidade de artigos
relacionados ao problema de tentar prever o uso de CPU é a de sistemas
distribuídos. Isso se deve ao fato de que em certos tipos de sistemas
distribuídos, é necessário que tarefas sejam despachadas para execução por
unidades de processamento que, preferencialmente, estejam ociosas~\cite{zhang2007cpu}.

Os trabalhos Dinda (2000, 2002)\nocite{dinda2000host}
\nocite{dinda2002evaluation} efetuam a coleta de informações de uso de
CPU em um conjunto formado por 38 computadores de diferentes tipos que
pertenciam ao \emph{Carnegie Mellon University} e ao \emph{Pittsburgh
Supercomputing Center}, nos Estados Unidos, pelo período de pouco mais que uma
semana. Com estas informações de uso de CPU foi feita uma análise deste conjunto
de dados e avaliou-se algumas técnicas de modelagem de séries temporais, e
sugerindo que seja usado o modelo AR(16) (\emph{Autoregressive model},
modelo auto-regressivo, com 16 coeficientes) para a predição de comportamento
de CPUs.

O modelo AR(16) consiste em uma representação na forma
$x_t = \phi_{1}x_{t-1} + \phi_{2}x_{t-2}+\dotsc,\phi_{p}x_{t-p}+e_t$, aonde a
$\phi_p$ correspondem aos parâmetros do modelo, sendo $p$ a ordem do mesmo, e
$e_t$ representa um ruído no modelo. Utilizando a técnica de Yule-Walker é
possível conhecer os parâmetros $\phi_p$ do modelo e assim prever itens na
série calculando-se $x_{t+1}$~\cite{baddour2005autoregressive}.
% FIXME rev Neusa, que modelo é Yule-Walker? Ao menos uma referência

O trabalho de~\citeonline{sodhi2008performance} apresenta uma técnica de
predição de comportamento de processos baseando-se na execução de uma pequena
aplicação, a qual tenta representar aspectos significativos de uma aplicação que
tem alto custo computacional de execução. A ferramenta apresentada coleta
informações a respeito do software que está sendo analisado e, então, gera um
“esqueleto” da mesma, o qual será executado em um ambiente que será
avaliado. Para avaliação de desempenho foram usadas ferramentas do conjunto NAS
Benchmark, com esqueletos de tempo de execução variando de 0,5 a 10 segundos.
Com exceção do esqueleto de 0,5 segundos, os outros tiveram uma taxa de erro de
predição média de 6,7\%. 

\citeonline{wolski2000predicting} compara duas medidas amplamente disponíveis
em sistemas Unix para avaliação de uso de CPU, a \emph{vmstat} que provê
informações de tempo gasto com processamento dentro de \emph{kernel} nos
aplicativos de usuário e a medida \emph{load average}, que é a contagem do
número de processos que estavam na fila de execução do escalonador no último
minuto. Foram feitas predições utilizando o \emph{Network Weather Service}
\cite{wolski1999network}, que é um conjunto de ferramentas distribuído para a
previsão de séries temporais, para a avaliação dessas medidas. As avaliações
demonstraram que ambas tem baixo erro para previsões de curto (10 segundos) e
médio (5 minutos) prazo.

O trabalho de \citeonline{dodonov2006adaptive} apresenta uma metodologia para a
previsão do comportamento de processo em ambiente de computação distribuída, de
maneira a permitir que o escalonador de tal ambiente possa saber quando as
aplicações irão precisar acessar dados que não estão próximos, de maneira a
trazê-las para unidades de processamento que estejam acessíveis com menor
latência. Utilizando a ferramenta \emph{GridBox}~\cite{dodonov2004gridbox} os
processos do ambiente distribuído têm suas chamadas de sistema e padrões de
acesso a rede observados e agrupados por meio de uma técnica de agrupamento
originalmente aplicada a textos~\cite{de2005automatic}, reduzindo a
dimensionalidade. Após isso são utilizadas as redes TDNN (\emph{Time-delay
Neural Network}) para o treinamento e predição do comportamento dos processos.

\section{Quadro comparativo de trabalhos relacionados}

\begin{table}
\centering
\hspace{-2cm} % FIXME arrumar no template
\quadro{Comparativo entre trabalhos relacionados}\label{quadro:relacionados}
\begin{tabular}{| m{5cm} | c | c | c |}
\hline
Técnica & Distrbuído & Externo & Políticas de migração \\
\hline
Adaptive technique for automatic communication access pattern discovery applied
to data prefetching in distributed applications using neural networks and
stochastic models \cite{dodonov2006adaptive} & Sim & Sim & Sim \\
\hline
Host Load Prediction Using Linear Models \cite{dinda2000host}
e \cite{dinda2002evaluation} & Não & Sim & Não \\
\hline
Prediction the CPU Availability Of Time-shared Unix System on The Computational
Grid \cite{wolski2000predicting} & Não & Sim & Não \\
\hline
Performance Prediction With Skeletons \cite{sodhi2008performance} & Sim & Não & Não \\
\hline

\end{tabular}
\end{table}

Todos os trabalhos descritos no quadro \ref{quadro:relacionados} tratam de
predição, assim é desnecessária uma coluna dedicada para indicar este aspecto.
A coluna Externo indica se a leitura de informações de CPU pode ser feita
externamente, sem executar software dentro do sistema observado, o que permite
a aplicação em sistemas de virtualização, atendendo a proposta deste trabalho.
A coluna Políticas de migração indica se o trabalho discute políticas que podem
ser usadas para migrar processos entre uma máquina e outra, o que é aplicável à
proposta do trabalho de migrar máquinas virtuais entre computadores diferentes.

% tese do eugeni
% dinda (análise de comportamento dos hosts, virtuoso)
% skeletons
% homeostatic,
% etc
