\documentclass[espaco=emeio,appendix=Name]{abnt}
\usepackage[utf8]{inputenc}
\usepackage[brazil]{babel}
\usepackage{hyperref}
\usepackage[alf]{abntcite}
\usepackage{mdwlist}
\usepackage{dsfont}

%% informações sobre o trabalho
\autor{Bogdano Arendartchuk}
\titulo{Analisador de Pascal reduzido - Manual do programador}
\comentario{Manual de funcionamento e especificação da ferramenta de
análise sintática para a linguagem Pascal reduzida.}
%%\instituicao{Universidade Tuiuti do Paraná}
\local{Curitiba}
\data{Junho de 2008}
\begin{document}
\capa
\folhaderosto
\sumario

\chapter{Introdução}

O objetivo deste texto é contruir um template para a confecção de
relatórios acadêmicos que sejam um utilizados na Universidade Tuiuti do
Paraná, utilizando a classe ABNTEX.

\chapter{Prosseguindo com o modelo}

Caros amigos, a constante divulgação das informações nos obriga à análise
das condições financeiras e administrativas exigidas. O cuidado em
identificar pontos críticos na crescente influência da mídia não pode mais
se dissociar dos modos de operação convencionais. A certificação de
metodologias que nos auxiliam a lidar com a contínua expansão de nossa
atividade faz parte de um processo de gerenciamento das diversas correntes
de pensamento. No entanto, não podemos esquecer que a consulta aos diversos
militantes auxilia a preparação e a composição das posturas dos órgãos
dirigentes com relação às suas atribuições. 

Por outro lado, a hegemonia do ambiente político assume importantes
posições no estabelecimento do investimento em reciclagem técnica. Ainda
assim, existem dúvidas a respeito de como o entendimento das metas
propostas talvez venha a ressaltar a relatividade da gestão inovadora da
qual fazemos parte. O empenho em analisar a adoção de políticas
descentralizadoras apresenta tendências no sentido de aprovar a manutenção
do sistema de participação geral. Pensando mais a longo prazo, o fenômeno
da Internet estende o alcance e a importância do retorno esperado a longo
prazo. 

\section{Os desafios para um modelo globalizado}

Todavia, o comprometimento entre as equipes desafia a capacidade de
equalização das diretrizes de desenvolvimento para o futuro. Por
conseguinte, a competitividade nas transações comerciais agrega valor ao
estabelecimento das condições inegavelmente apropriadas. O que temos que
ter sempre em mente é que a necessidade de renovação processual é uma das
consequências de alternativas às soluções ortodoxas. Percebemos, cada vez
mais, que a estrutura atual da organização acarreta um processo de
reformulação e modernização dos procedimentos normalmente adotados.

Acima de tudo, é fundamental ressaltar que a valorização de fatores
subjetivos exige a precisão e a definição do remanejamento dos quadros
funcionais. É claro que a complexidade dos estudos efetuados pode nos levar
a considerar a reestruturação dos níveis de motivação departamental. A
prática cotidiana prova que a percepção das dificuldades representa uma
abertura para a melhoria das formas de ação.

\subsection{Desafios imediatos}

A certificação de metodologias que nos auxiliam a lidar com a contínua
expansão de nossa atividade aponta para a melhoria do fluxo de informações.
Pensando mais a longo prazo, a consulta aos diversos militantes garante a
contribuição de um grupo importante na determinação das posturas dos órgãos
dirigentes com relação às suas atribuições. Todas estas questões,
devidamente ponderadas, levantam dúvidas sobre se o início da atividade
geral de formação de atitudes talvez venha a ressaltar a relatividade das
diversas correntes de pensamento. Assim mesmo, o desenvolvimento contínuo
de distintas formas de atuação afeta positivamente a correta previsão das
direções preferenciais no sentido do progresso.

No mundo atual, a hegemonia do ambiente político estende o alcance e a
importância do investimento em reciclagem técnica. Caros amigos, o consenso
sobre a necessidade de qualificação maximiza as possibilidades por conta
dos relacionamentos verticais entre as hierarquias. A prática cotidiana
prova que a estrutura atual da organização oferece uma interessante
oportunidade para verificação do sistema de formação de quadros que
corresponde às necessidades. O incentivo ao avanço tecnológico, assim como
o aumento do diálogo entre os diferentes setores produtivos cumpre um papel
essencial na formulação do processo de comunicação como um todo.

\end{document}
