%% vim:ft=tex
\documentclass[chapter=TITLE,section=Title,espaco=umemeio,tocpage=plain,appendix=Name]{abnt}
\usepackage{hyperref}
\usepackage[utf8]{inputenc}
\usepackage[brazil]{babel}
\usepackage[alf]{abntcite}
\usepackage{mdwlist}
\usepackage{dsfont}

%% xunxo especial para seguir as normas da UTP
\usepackage{xunxos-utp}

%% informações sobre o trabalho
\autor{Nicolai Nicolaiev}
\titulo{Eis um título para o trabalho}
\comentario{Trabalho da disciplina de Seiláoquê do Curso de Ciência da
Computação da Universidade Tuitui do Paraná. }
\instituicao{Universidade Tuiuti do Paraná}
\orientador[Professor: ]{Bla}
\local{Curitiba}
\data{2010}

\begin{document}

\citeoption{minhasopcoes}

%%\capa
%%\folhaderosto
\UTPCapa

\sumario

\chapter{INTRODUÇÃO}

As experiências acumuladas demonstram que a contínua expansão de nossa
atividade afeta positivamente a correta previsão das novas proposições. Do
mesmo modo, a valorização de fatores subjetivos causa impacto indireto na
reavaliação dos paradigmas corporativos. Percebemos, cada vez mais, que o
fenômeno da Internet ainda não demonstrou convincentemente que vai
participar na mudança das diretrizes de desenvolvimento para o futuro. A
prática cotidiana prova que a determinação clara de objetivos talvez venha
a ressaltar a relatividade das formas de ação. 


\chapter{OUTRA COISA}

Segundo ~\cite{joachims1998text}, a natureza do aprendizado estatístico não
deve-se apenas aos problemas da humanidade.

Porém, segundo ~\cite{vapnik2000nature}, nada disso é verdade.

Pensando mais a longo prazo, a expansão dos mercados mundiais pode nos
levar a considerar a reestruturação de todos os recursos funcionais
envolvidos. Acima de tudo, é fundamental ressaltar que o aumento do diálogo
entre os diferentes setores produtivos auxilia a preparação e a composição
das diversas correntes de pensamento. Todavia, a execução dos pontos do
programa acarreta um processo de reformulação e modernização da gestão
inovadora da qual fazemos parte. Ainda assim, existem dúvidas a respeito de
como a necessidade de renovação processual prepara-nos para enfrentar
situações atípicas decorrentes dos modos de operação convencionais. A nível
organizacional, a competitividade nas transações comerciais exige a
precisão e a definição de alternativas às soluções ortodoxas.

E em imagens, a generalização para duas dimensões para uma matriz $p$ de
tamanho $n \times n$:
\begin{eqnarray}
\nonumber
G_{ij} = \frac{1}{\sqrt{2n}} C_i C_j \sum_{x=0}^{n-1} \sum_{y=0}^{n-1}
         p_{xy} \cos{ \left ( \frac{(2y + 1) j \pi}{2n} \right ) }
              \cos{ \left ( \frac{(2x + 1) i \pi}{2n} \right ) }
\end{eqnarray}

\section{PUTS AÍ É MUITO TOSCO FAZER ASSIM}

Foo bar baz.

\subsection{Mais alguma coisa}

 O incentivo ao avanço tecnológico, assim como a hegemonia do ambiente
político não pode mais se dissociar das condições inegavelmente
apropriadas. Gostaria de enfatizar que a constante divulgação das
informações oferece uma interessante oportunidade para verificação dos
relacionamentos verticais entre as hierarquias. O empenho em analisar a
mobilidade dos capitais internacionais é uma das consequências dos métodos
utilizados na avaliação de resultados.

\subsubsection{outra coisa finalmente}

Caros amigos, a crescente influência da mídia nos obriga à análise dos
procedimentos normalmente adotados. A certificação de metodologias que nos
auxiliam a lidar com o início da atividade geral de formação de atitudes
estende o alcance e a importância do retorno esperado a longo prazo. Assim
mesmo, a complexidade dos estudos efetuados assume importantes posições no
estabelecimento do sistema de participação geral.

\bibliography{biblio}

\end{document}
