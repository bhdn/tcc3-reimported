\chapter{Testes e observações}\label{sec:resultados}

\section{Coleta dos dados}

O dados foram coletados de cinco máquinas do ambiente de compilação do
projeto Mandriva Linux entre os dias 15 de março de 2011 e 11 de abril de
2011, com medidas de uso tempo de uso de CPU sendo realizadas a cada 30
segundos, resultando num total de 80200 valores para cada máquina.

Todas as máquinas possuem 4 CPUs e \emph{clock} de 2800MHz. As máquinas n2,
n3 e n4 são utilizadas exclusivamente para execução de software para
arquitetura i686 (32 bits,) enquanto que n6 e seggie são utilizadas para
execução e compilção de software em arquitetura AMD64 (64 bits.)

\section{Caracterização}

Mesmo estas máquinas fazendo parte de um ambiente de compilação que é
utilizado diariamente para compilar vários pacotes de software do projeto
Mandriva Linux, 45\% dos valores de uso de CPU lidos foram 0\%, ou seja,
as máquinas estiveram ociosas durante 45\% do tempo durante os 27 dias de
coleta. A figura \ref{fig:prehistn4} mostra a distribuição para uma das
máquinas do ambiente. Os histogramas das outras máquinas estão no apêndice
\ref{chap:histogramas}, a partir da página \pageref{chap:histogramas}.

\begin{figure}[htp]
\centering
\includegraphics{src/test-data/workload/histograma-n4.pdf}
\figinfo{Distribuição dos valores de uso de CPU da máquina n4}
\label{fig:prehistn4}
\end{figure}


A figura~\ref{fig:predispn4} mostra a dispersão dos valores de CPU durante
todo o período de observação, enquanto que a figura~\ref{fig:disp4hn4}
mostra o o comportamento durante um período de quatro horas, aonde é
possível observar mais claramente cada ponto observado. Nota-se que a
transição entre os valores é pouco suave. O valor do desvio padrão para
cada máquina está disposto no quadro \ref{quadro:desviopadrao}.

\begin{table}[htp]
\centering
\hspace{-2cm} % FIXME arrumar no template
\quadro{Média e desvio padrão das máquinas observadas}\label{quadro:desviopadrao}
\begin{tabular}{| c | c | c | c |}
\hline
Máquina & Média & Desvio padrão \\
\hline
n2 	& 	24,5623722901 & 34,45394416 \\
n3 	& 	6,83392723648 & 18,5214665282 \\
n4 	& 	27,7236937184 & 32,1217945295 \\
n6 	& 	35,1048768939 & 37,8408009198 \\
seggie 	&	19,9159114989 & 32,068825708 \\
\hline
\end{tabular}
\end{table}

\begin{figure}
\centering
\includegraphics[width=0.7\textwidth]{src/test-data/workload/dispersion-n4.png}
\figinfo{Representação dos valores de uso de CPU da máquina n4}
\label{fig:predispn4}
\end{figure}

Nesta mesma figura~\ref{fig:disp4hn4} é possível observar o ciclo de vida
de uma tarefa de compilação de um pacote de software. A partir do ponto 150
nota-se um crescimento moderado no uso de CPU, que pode ser associado ao
estágio de preparação que é executado antes da compilação em si. Este
estágio executa a descompressão dos códigos-fonte que serão compilados,
geralmente usando os softwares gzip\footnote{http://www.gzip.org} e
bzip2\footnote{http://bzip2.org}. Tal descompressão é uma tarefa faz uso
intenso de leitura e escrita de disco e também de CPU, porém estas
ferramentas fazem uso apenas de uma CPU, resultando em leituras de CPU em
torno de 25\%.

Após o estágio de preparação o software é compilado. A maioria dos
códigos-fonte é preparada para compilar código parelelamente, permitindo
assim fazer uso de todas as CPUs disponíveis, chegando aos 100\% de
utilização. Finalmente, quando a tarefa de compilação é concluída os
binários gerados são comprimidos em um arquivo, resultando novamente no uso
moderado de CPU.

\begin{figure}[htp]
\centering
\includegraphics[width=0.7\textwidth]{src/test-data/workload/dispersion-4h-n4.png}
\figinfo{Representação dos valores de uso de CPU da máquina n4 durante um
período de 4 horas}
\label{fig:disp4hn4}
\end{figure}

É possível observar também na figura \ref{fig:predispn4}, e também nas
figuras \ref{fig:appdispn2}, \ref{fig:appdispn6} e \ref{fig:appdispseggie} no
apêndice, que existe um período de intenso uso de CPU em torno do ponto
30000. Isso deve-se a uma extensa tarefa de recompilação que durou
aproximadamente três dias e utilizou várias máquinas do ambiente de
compilação.

\section{Predição}


