\chapter{Testes e observações}\label{sec:resultados}

\section{Coleta dos dados}

O dados foram coletados de cinco máquinas do ambiente de compilação do
projeto Mandriva Linux entre os dias 15 de março de 2011 e 11 de abril de
2011, com medidas de uso tempo de uso de CPU sendo realizadas a cada 30
segundos, resultando num total de 80~200 valores para cada máquina.

Todas as máquinas possuem 4 CPUs e \emph{clock} de 2~800MHz. As máquinas n2,
n3 e n4 são utilizadas exclusivamente para execução de software para
arquitetura i686 (32 bits,) enquanto que n6 e seggie são utilizadas para
execução e compilção de software em arquitetura AMD64 (64 bits.)

\section{Caracterização}

Mesmo estas máquinas fazendo parte de um ambiente de compilação que é
utilizado diariamente para compilar vários pacotes de software do projeto
Mandriva Linux, 45\% dos valores de uso de CPU lidos foram 0\%, ou seja,
as máquinas estiveram ociosas durante 45\% do tempo durante os 27 dias de
coleta. A figura \ref{fig:prehistn4} mostra a distribuição para uma das
máquinas do ambiente. Os histogramas das outras máquinas estão no apêndice
\ref{chap:histogramas}, a partir da página \pageref{chap:histogramas}.

\begin{figure}[htp]
\centering
\includegraphics{src/test-data/workload/histograma-n4.pdf}
\figinfo{Distribuição dos valores de uso de CPU da máquina n4}
\label{fig:prehistn4}
\end{figure}


A figura~\ref{fig:predispn4} mostra a dispersão dos valores de CPU durante
todo o período de observação, enquanto que a figura~\ref{fig:disp4hn4}
mostra o o comportamento durante um período de quatro horas, aonde é
possível observar mais claramente cada ponto observado. Nota-se que a
transição entre os valores é pouco suave. O valor do desvio padrão para
cada máquina está disposto no quadro \ref{quadro:desviopadrao}.

\begin{table}[htp]
\centering
\hspace{-2cm} % FIXME arrumar no template
\quadro{Média e desvio padrão das máquinas observadas}\label{quadro:desviopadrao}
\begin{tabular}{| c | c | c | c |}
\hline
Máquina & Média & Desvio padrão \\
\hline
n2 	& 	24,5623722901 & 34,45394416 \\
n3 	& 	6,83392723648 & 18,5214665282 \\
n4 	& 	27,7236937184 & 32,1217945295 \\
n6 	& 	35,1048768939 & 37,8408009198 \\
seggie 	&	19,9159114989 & 32,068825708 \\
\hline
\end{tabular}
\end{table}

\begin{figure}
\centering
\includegraphics[width=0.7\textwidth]{src/test-data/workload/dispersion-n4.png}
\figinfo{Representação dos valores de uso de CPU da máquina n4}
\label{fig:predispn4}
\end{figure}

Nesta mesma figura~\ref{fig:disp4hn4} é possível observar o ciclo de vida
de uma tarefa de compilação de um pacote de software. A partir do ponto 150
nota-se um crescimento moderado no uso de CPU, que pode ser associado ao
estágio de preparação que é executado antes da compilação em si. Este
estágio executa a descompressão dos códigos-fonte que serão compilados,
geralmente usando os softwares gzip\footnote{http://www.gzip.org} e
bzip2\footnote{http://bzip2.org}. Tal descompressão é uma tarefa faz uso
intenso de leitura e escrita de disco e também de CPU, porém estas
ferramentas fazem uso apenas de uma CPU, resultando em leituras de CPU em
torno de 25\%.

Após o estágio de preparação o software é compilado. A maioria dos
códigos-fonte é preparada para compilar código parelelamente, permitindo
assim fazer uso de todas as CPUs disponíveis, chegando aos 100\% de
utilização. Finalmente, quando a tarefa de compilação é concluída os
binários gerados são comprimidos em um arquivo, resultando novamente no uso
moderado de CPU.

\begin{figure}[htp]
\centering
\includegraphics[width=0.7\textwidth]{src/test-data/workload/dispersion-4h-n4.png}
\figinfo{Representação dos valores de uso de CPU da máquina n4 durante um
período de 4 horas}
\label{fig:disp4hn4}
\end{figure}

É possível observar também na figura \ref{fig:predispn4}, e também nas
figuras \ref{fig:appdispn2}, \ref{fig:appdispn6} e \ref{fig:appdispseggie} no
apêndice, que existe um período de intenso uso de CPU em torno do ponto
30~000. Isso deve-se a uma extensa tarefa de recompilação que durou
aproximadamente três dias e utilizou várias máquinas do ambiente de
compilação.

\section{Predição}

O primeiro passo ao avaliar as técnicas de predição foi ajustar os
parâmetros que eram utilizados pelas mesmas para o domínio de dados que é
utilizado neste projeto. Para o caso de $k$-NN, o parâmetro investigado é o
número de vizinhos ($k$), e no caso de SVM, $C$ e $\sigma$, descritos a
partir da página \pageref{sec:margenssuaves}. Além disso, há outros dois
parâmetros que são específicos a este projeto e são relacionados à etapa de
predição: o tamanho da janela utilizada para treinamento e o tamanho da
janela utilizada para definir o uso de CPU futuro.

Decidiu-se por adotar a seguinte estratégia ao escolher os parâmetros:
primeiro foram utilizados para os parâmetros de valores futuros e tamanho
de janela valores que podem ser considerados “ideais” para o problema deste
projeto. Tendo esses parâmetros fixos, foi possível fazer a busca pelos
melhores parâmetros para $k$-NN e SVM. Então, tendo os parâmetros
específicos de cada técnica, foram feitos testes com diferentes tamanhos de
janela para treinamento e futuro.

% me dei conta agora que a definição de “futuro” deveria estar associada à
% influência que realmente o valor vai ter no vetor de características

O tamanho de janela considerado ideal para este projeto foi 10, ou seja,
representa um horizonte de 5 minutos. Este período de tempo é suficiente
para incluir o tempo de migração das máquinas virtuais (abordado a seguir.)

Para os testes de $k$ de $k$-NN, foi utilizado um programa escrito em
\emph{shell script} que executa a etapa de treinamento e predição para
valores de $k$ que vão de 2 até 20 e também para 30, 40, 50, 75, 100 e 150
vizinhos. O script utilizado está no apêndice \ref{chap:programasteste}. A
figura \ref{fig:XXXX} mostra a taxa de acerto de acordo com o valor de $k$
para a máquina n4.

%%%XXX figura de knn aqui!!

Para o caso de SVM, foram utilizadas os parâmetros sugeridos por
\citeonline{hsu2003practical}, com valores de $C$ indo de $2^{-5}$ até
$2^9$ e $\sigma$ de $^{-15}$ até $2^{3}$. A figura \ref{fig:XXXX} apresenta
o desempenho desses parâmetros para utilizando dados da máquina n4. Após
isso, foi executada mais uma rodada de testes, desta vez utilizando uma
faixa de valores mais curta, indo de 0,3125 até 2, tando para $C$ quanto
para $\sigma$. O script utilizado para testes está no apêndice
\ref{chap:programasteste}.

%%% XXX XXX XXX CROSS VALIDATION!!!
%%% XXX figura de parâmetros da SVM aqui!

Finalmente, foram feitos os testes utilizando diferentes tamanhos de janela
para trainamento e valores futuros. A figura \ref{fig:XXXX} apresenta o
espaço de busca e o desempenho utilizando dados da máquina n4.

\subsection{Classificação cruzada}

\begin{table}[htp]
\centering
\hspace{-2cm} % FIXME arrumar no template
\quadro{Desempenho de classificação cruzada, usando SVM}\label{quadro:clacruzada}
\begin{tabular}{| l | c | c | c | c | c | c |}
\hline
\multirow{2}{*}{Máq. classificada} & \multicolumn{6}{c|}{Máquina treinada} \\
\cline{2-7}
		& n2      & n3       & n4      	 & n6        & seggie  	 & média   \\
\hline
n2      	& 99,39\%  & 98,59\% &   99,51\% &   96,10\% &   96,05\% & 97,93\% \\
n3      	& 99,82\%  & 99,78\% &   99,91\% &   99,09\% &   99,08\% & 99,54\% \\
n4      	& 99,39\%  & 99,11\% &   99,68\% &   98,50\% &   98,52\% & 99,04\% \\
seggie  	& 98,73\%  & 98,06\% &   99,09\% &   99,20\% &   99,38\% & 98,89\% \\
n6      	& 99,13\%  & 98,30\% &   99,27\% &   99,34\% &   99,36\% & 99,08\% \\
\hline
\end{tabular}
\end{table}

Para o caso de máquinas desconhecidas, como descrito na seção
\ref{sec:maquinasdesconhecidas}, o software tentará recorrer ao histórico
de uma máquina especial, chamada \emph{generic}. Assim, para o grupo de
máquinas estudado neste projeto, foi feito um teste em que todas as
máquinas tiveram seus históricos preditos por todas as outras máquinas, de
maneira a identificar qual delas tem melhor capacidade de generalização
para o ambiente observado. No quadro \ref{quadro:clacruzada}, que apresenta
os desempenhos obtidos, é possível observar que a máquina n4 possui o
melhor desempenho de classficação.

\section{Testes de virtualização}

\subsection{Ambiente de testes}

O ambiente utilizado para testes de virtualização é composto por três máquinas
hospedeiras, uma tendo processador Intel Core i3 550, \emph{clock} de 3GHz e 4
gigabytes de memória RAM, e as outras duas máquinas com processador Intel
Pentium Dual-Core, \emph{clock} de 3GHz e 4 gigabytes de memória RAM. A rede de
dados utilizada para conectar estas máquinas era do tipo 100BASE-TX, que pode
transmitir dados a 100 megabits por segundo\footnote{http://standards.ieee.org/about/get/802/802.3.html}.

As máquinas virtuais eram instalações mínimas de Mandriva Linux, com imagens de
disco ocupando 1 gigabyte, emulando máquinas com processador i686 e 128
megabytes de RAM.

\subsection{Migração}

Para conhecer o tempo esperado de uma migração entre uma máquina hospedeira e
outra, foram feitas 50 migrações de máquinas virtuais, em ambos os sentidos.
Observou-se que o tempo médio de migração foi de 8,9 segundos, com desvio padrão
de 0,8 segundos.

Deve-se levar em consideração que as máquinas virtuais utlizadas para este e
outros testes no trabalho têm apenas 128MB de memória RAM, o que é uma
quantidade de memória muito pequena para os padrões de uso atuais. Porém,
espera-se que em casos aonde as máquinas virtuais tenham mais memória, seja
utilizada uma rede do tipo \emph{gigabit} (1000BASE-TX), que permite ter taxas
de transmissão pelo menos dez vezes mais rápidas.

\subsection{Simulação de carga}\label{sec:loadsim}

Para que seja possível avaliar o produto final deste projeto, ou seja, a
consolidação de máquinas virtuais, foi necessário avaliar também a técnica
utilizada para reproduzir o comportamento de uso de CPU de máquinas virtuais de
acordo com um histórico de outra máquina, descrita na seção \ref{sec:desemp}.

Para tal, foi implementado um programa em linguage Python que utiliza \emph{busy
loops} para simular a carga de CPU. Seu código fonte está no apêndice, página
\pageref{fig:progsimload}.

A figura \ref{fig:simworkload} apresenta o uso de CPU medido da máquina virtual
sobreposto com o histórico que foi obtido da máquina n4. Pode-ser observar que,
mesmo com alguns \emph{outliers}, a maioria dos valores foi corretamente
simulada.

%%% FIXME colocar a figura aqui!

\section{Consolidação}

Para a etapa de consolidação, foram feitos dois tipos de testes. Um simulado, em
que históricos de máquinas são utilizados para fornecer informações de uso de
CPU ao software de consolidação que faz parte deste projeto. O outro tipo de
teste consiste em simular a carga em máquinas virtuais reais, utilizando a
biblioteca \libvirt{} para acessar as informações de uso de CPU.

\subsection{Carga simulada}




\section{Aspectos não explorados neste trabalho}

% - a definição de valor “futuro”, talvez escolher o valor máximo faria
%   mais sentido.
