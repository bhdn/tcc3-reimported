%
% Introdução
%

\chapter{Introdução}

Com a crescente disponibilização de aplicações por meio da Internet, ter
recursos computacionais (processamento, armazenamento, entre outros)
resilientes para prover as mesmas é cada vez mais um requisito. Assim,
empresas que oferecem estes recursos como forma de serviço, geralmente
empresas de hospedagem de \emph{websites} e aplicações, precisam adotar
tecnologias de alta disponibilidade em seus ambientes computacionais.

Neste contexto, as tecnologias de virtualização possibilitam que um ou mais
sistemas computacionais sejam executados dentro de um outro sistema,
permitindo, assim, que o estado destes possa ser controlado por uma aplicação
externa, chamada também de hipervisor, que pode prover serviços de
alta disponibilidade desses sistemas, comumente chamados de máquinas
virtuais. Pode-se, por exemplo, migrar uma máquina virtual que está
executando em um computador para outro ou mesmo mudar sua configuração,
como quantidade de memória RAM, sem que este processo resulte em uma
interrupção prejudicial ao serviço disponibilizado.

Produtos como \emph{VMware,}\footnote{http://www.vmware.com}
\emph{Xen,}\footnote{http://xensource.com/}
\emph{Virtualbox}\footnote{http://virtualbox.org/} e
\emph{KVM}\footnote{http://linux-kvm.org/} (desenvolvidos por VMware,
Citrix, Oracle e Red Hat, respectivamente) oferecem ferramentas para a
execução e manutenção de máquinas virtuais. Além disso, esses produtos
proveem APIs (\emph{Application Programming Interface}, Interface para Programação
de Aplicações) que permitem que terceiros desenvolvam aplicações
específicas que gerenciem essas máquinas virtuais ou coletem informações a
respeito das mesmas. Para abstrair as diferenças entre cada API de cada
fornecedor, a biblioteca \libvirt{} oferece uma API uniforme para gerenciar
máquinas virtuais independente de qual produto de virtualização é
utilizado.

Tendo alta disponibilidade, esses provedores de recursos computacionais
precisam, também, tentar fazer bom uso de seu parque de computadores
disponíveis, minimizando custos de manutenção e energia elétrica
sem degradar a qualidade do que é provido aos clientes. Mantendo mais
máquinas virtuais em um mesmo hipervisor diminui custos de energia e
manutenção, tendo, porém o risco de degradar o desempenho, uma vez que
unidades de processamento (CPUs) e barramentos de dados são utilizadas
concorrentemente entre as máquinas virtuais.

Com isso, fica evidente o potencial de aproveitar as ferramentas e APIs de
virtualização para observar o comportamento de máquinas virtuais de maneira
a tentar prever quando é mais adequado separar máquinas virtuais ou
agrupá-las\footnote{Entende-se agrupamento simplesmente por
executar máquinas virtuais em um mesmo computador, independente do software
que está executando dentro dessas máquinas. Porém, o caso mais comum em
ambientes comerciais trata-se de servidores de aplicações web e banco de
dados.} em um mesmo computador. 

Assim, este trabalho apresenta um software que utiliza técnicas de
aprendizado de máquina para prever o comportamento de máquinas
virtuais em um ambiente de virtualização baseado em \libvirt{}, tendo como
objetivo a minimização (consolidação) do uso de recursos usados em parque
computacional, quando possível. Serão avaliadas as técnicas de Máquinas de
Vetores de Suporte e $k$-NN para a predição e a técnica de \emph{first fit} para
a alocação. Para avaliar a metodologia será utilizado um histórico de uso de CPU
em um ambiente de compilação de software que faz parte do projeto \emph{Mandriva
Linux\footnote{http://www.mandriva.com/}}.

Na seção \ref{sec:aprendizado} é apresentado o conceito de aprendizado de
máquina, com uma descrição detalhada de Máquinas de Vetores de Suporte na seção
\ref{sec:svm}. A seção \ref{sec:relacionados} apresenta trabalhos com objetivos
e/ou técnicas relacionadas.Os conceitos relacionados às tecnologias de
virtualização são descritos na seção \ref{sec:virt}, com uma descrição à
ferramenta \libvirt{} na seção \ref{sec:libvirt}. Então é apresentada a
metodologia de desenvolvimento do projeto na seção \ref{sec:meto}. Finalmente,
na seção \ref{sec:resultados} são apresentados os testes realizados e resultados
obtidos.
