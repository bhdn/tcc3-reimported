%
% Introdução
%

\chapter{Introdução}

Com a crescente disponibilização de aplicações por meio da Internet, ter
recursos computacionais (processamento, armazenamento, etc) resilientes
para prover as mesmas é cada vez mais um requisito. Assim, empresas que
oferecem esses recursos como forma de serviço (geralmente empresas de
hosting de websites e aplicações) precisam adotar tecnologias de
alta-disponibilidade em seus ambientes computacionais.

Nesse contexto, as tecnologias de virtualização possibilitam que um ou mais
sistemas computacionais sejam executados dentro de um outro sistema,
permitindo assim que o estado destes possa ser controlado por uma aplicação
externa, chamada também de hypervisor, que pode prover serviços de
alta-disponibilidade desses sistemas (geralmente chamados de máquinas
virtuais). Pode-se, por exemplo, migrar uma máquina virtual que está
executando em um computador para outro ou mesmo mudar sua configuração,
como quantidade de memória RAM, sem que este processo resulte na
interrupção do serviço disponibilizado.

Produtos como VMWare\footnote{http://www.vmware.com},
Xen\footnote{http://xensource.com/},
Virtualbox\footnote{http://virtualbox.org/} e
KVM\footnote{http://linux-kvm.org/} (desenvolvidos por VMWare,
Citrix, Oracle e Red Hat, respectivamente) oferecem ferramentas para a
execução e manutenção de máquinas virtuais. Além disso, esses produtos
proveem APIs (Application Programming Interface, Interface para Programação
de Aplicações) que permitem que terceiros desenvolvam aplicações
específicas que gerenciem essas máquinas virtuais ou coletem informações a
respeito das mesmas. Para abstrair as diferenças entre cada API de cada
fornecedor, a biblioteca libVirt oferece uma API uniforme para gerenciar
máquinas virtuais independente de qual produto de virtualização é
utilizado.

Tendo alta-disponibilidade, esses provedores de recursos computacionais
precisam também tentar fazer bom uso de seu parque de computadores
disponíveis de maneira a  minimizar custos de manutenção e energia elétrica
sem degradar a qualidade do que é provido aos clientes. Mantendo mais
máquinas virtuais em um mesmo hypervisor diminui custos de energia e
manutenção porém ao risco de degradar o desempenho pois as unidades de
processamento (CPUs) e barramentos de dados são utilizadas concorrentemente
entre as máquinas virtuais.

Com isso, fica evidente o potencial de aproveitar as ferramentas e APIs de
virtualização para observar o comportamento de máquinas virtuais de maneira
a tentar prever quando é mais adequado separar máquinas virtuais ou
agrupá-las\footnote{Entende-se agrupamento simplesmente por
executar máquinas virtuais em um mesmo computador, independente do software
que está executando dentro dessas máquinas. Porém, o caso mais comum em
ambientes comerciais trata-se de servidores de aplicações web e banco de
dados.} em um mesmo computador. 

Para a análise e predição do comportamento das máquinas virtuais, é
possível utilizar os estudos das áreas de sistemas distribuídos e sistemas
operacionais que, por exemplo, tentam prever o comportamento de processos
para tentar deixar dados que potencialmente serão utilizados em
dispositivos de memória de mais baixa latência.
