\chapter{Conclusão}

Pode-se considerar que o objetivo deste trabalho foi parcialmente atingido. Foi
possível predizer o uso de CPU no ambiente estudado com taxas de acerto a
partir de 98\% utilizando a técnica de máquinas de vetores de suporte com tempo
de execução sem degradar o sistema como um todo.

Além disso, foi possível demonstrar que é possível implementar um software de
consolidação utilizando as técnicas de predição avaliadas e avaliar seu
desempenho através de um simulador. Observou-se que um ambiente com cinco
máquinas reais teve sua carga simulada utilizando, em grande parte do tempo,
apenas duas máquinas hospedeiras

Houve dificuldade, porém, em avaliar o desempenho da ferramenta de consolidação
em um ambiente real, em função da dificuldade de obter uma fonte de informação
apropriada para conhecer quanto tempo de CPU uma máquina virtual realmente
demanda.

\section{Trabalhos futuros}

É possível modificar cada componente distinto do projeto, como apresentado
na seção \ref{sec:modelagem}, para avaliar diferentes abordagens para
implementá-los. Também é possível que cada funcionalidade seja separada e
avaliada como um problema distinto.

Para as etapas de coleta de dados e predição, é possível considerar o uso
de mais informações providas pelo comando \emph{vmstat} como, por exemplo,
a quantidade de dados lida de dispositivos de bloco ou o número de
processos em espera para execução.

Especificamente para a etapa de predição, é possível comparar o desempenho
de SVM e $k$-NN com outras técnicas para predição de séries temporais.

Para a etapa de consolidação das máquinas virtuais, pode-se comparar o
desempenho com as técnicas avaliadas no trabalho de
\citeonline{Ferreto20111027}, bem como os históricos de carga que são
utilizados no mesmo.

Finalmente, é possível avaliar o uso de outras técnicas para a coleta de
demanda de uso de CPU, de maneira a evitar o quando adiciona-se muitas
máquinas virtuais em um mesmo hospedeiro, como descrito na seção
\ref{sec:cargareal}. É possível avaliar a implementação de um módulo
\emph{virtio} que seria carregado pelo \emph{kernel} de cada máquina
virtual, e reportaria a demanda de uso de CPU para o monitor de máquinas
virtuais.

% Componentes que poderiam ser melhorados:
% - predição: técnicas de predição:tab
% - escalonador/provisionador/consolidador, testar algoritmos específicos, focar apenas nisso
% - coleta de dados: usar virtio para comunicar-se com o kernel etc
