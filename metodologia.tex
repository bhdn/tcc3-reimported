%
% Metodologia
%
%
% - diagramas de contexto descrevendo quais serão as informações de
%   entrada, qual módulo do software fará o processamento e como serão
%   disponibilizadas as informações da predição,
% - diagrama de classes para a implementação do software,
% 
% - descrição de quais técnicas serão utilizadas e como serão implementadas,
% - descrição de um protocolo para avaliação do desempenho das técnicas
% implementadas.
%
%

\chapter{Metodologia}\label{sec:meto}

\section{Introdução}

Em um ambiente de virtualização gerenciado pela ferramenta \emph{libVirt}
(apresentada na seção \ref{sec:libvirt}), o software executará em plano de
fundo (ou seja, será um \emph{daemon}) e estará periodicamente coletando
informações quantitativas a respeito do tempo de uso de processador de
cada uma das máquinas virtuais que fazem parte de um determinado
\emph{domínio}. O software utilizará estas informações para treinar uma
técnica de aprendizado de máquina. Utilizando esta técnica, o software
poderá então prever em quais situações alguma máquina virtual exigirá mais
uso de processador que a máquina \emph{host} em que ela executa terá
condições de atender sem prejudicar as outras.

\section{Modelagem do software}

\section{Treinamento}

\section{Avaliação de desempenho}\label{sec:desemp}

% utilizar SVM, K-NN, AR(16) (se for fácil de implementar)
% variar o tamanho das janelas de histórico
% utilizar uma instância do classificador para todas as VMs, versus
% utilizar uma instância nova para cada VM
