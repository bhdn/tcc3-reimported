%
% Metodologia
%
% - diagramas de contexto descrevendo quais serão as informações de
%   entrada, qual módulo do software fará o processamento e como serão
%   disponibilizadas as informações da predição,
% - diagrama de classes para a implementação do software,
% - descrição de quais técnicas serão utilizadas e como serão implementadas,
% - descrição de um protocolo para avaliação do desempenho das técnicas
% implementadas.
%

\chapter{Metodologia de desenvolvimento}\label{sec:meto}

\section{Introdução}

% FIXME rev Neusa: “Deves escrever algo para introduzir o capítulo, não começar
% falando da figura 3”
A figura \ref{fig:hostguests1} apresenta um ambiente de virtualização
mínimo, onde tem-se um computador, denominado \emph{host}, que está
hospedando três máquinas virtuais vm-1, vm-2 e
vm-3. Estas máquinas virtuais estão sendo gerenciadas por qualquer
uma das ferramentas que são suportadas pela \libvirt{}, e cada uma delas
pode estar executando qualquer sistema operacional, e nestes podem estar
executando qualquer tipo de software, mais comumente servidores de banco de
dados e aplicações Web.

\begin{figure}[htp]
\centering
\includegraphics{img-host-guests1.pdf}
\figinfo{Ambiente com um hospedeiro e poucos hóspedes}
\label{fig:hostguests1}
\end{figure}

Se o administrador deste ambiente esperava que as aplicações que
executam dentro das máquinas virtuais tenham um desempenho aceitável,
então ele conhece os usos de processador, memória e entrada/saída
dessas aplicações e dimensiona o ambiente de tal maneira que uma máquina
virtual não degrade o desempenho de outra a ponto de torná-lo inaceitável.

Porém, à medida que mais máquinas virtuais forem necessárias, como é
apresentado na figura \ref{fig:hostguests2}, novas máquinas físicas terão
que ser alocadas para acomodar as novas demandas. E a partir deste ponto
gerenciar este parque computacional começa a tornar-se uma tarefa
desafiadora, pois é necessário tentar acomodar tais máquinas virtuais
levando em conta as necessidades particulares de cada aplicação que está
executando neste ambiente.

\begin{figure}[htp]
\centering
\includegraphics{img-host-guests2.pdf}
\figinfo{Ambiente com vários hospedeiros e hóspedes}
\label{fig:hostguests2}
\end{figure}

Na figura \ref{fig:hostguests3}, a máquina host3 possui cinco máquinas
virtuais que ocupam, ao todo, 57\% do tempo de processamento. Se estas
máquinas virtuais mantiverem o mesmo comportamento no decorrer do tempo de
vida deste ambiente, a única mudança necessária seria de talvez desligar
host1 e migrar estas máquinas virtuais para host3, já que
este hospedeiro ainda tem aproximadamente 43\% de tempo de processamento
sendo desperdiçado. Sendo host1 desligada, haveria menos custos
com energia e manutenção.

\begin{figure}[htp]
\centering
\includegraphics{img-host-guests3.pdf}
\figinfo{Dois hospedeiros e carga de CPU aceitável das MVs}
\label{fig:hostguests3}
\end{figure}

Porém, os serviços que executam nestas máquinas virtuais podem sofrer um
abrupto aumento de demanda, o que é mais ainda provável em aplicações que
estão acessíveis por meio da Internet, como servidores web ou serviços
relacionados, como servidores de banco de dados. Nestes casos, o
administrador do ambiente precisa provisionar rapidamente recursos para
atender a essa demanda evitando assim degradar a qualidade da
disponibilização do serviço.

\begin{figure}[htp]
\centering
\includegraphics{img-host-guests4.pdf}
\figinfo{Dois hospedeiros e carga das MVs desbalanceada entre eles}
\label{fig:hostguests4}
\end{figure}

No exemplo da figura \ref{fig:hostguests4}, a máquina virtual
vm-11 está executando algum processo que está exigindo muito tempo
de CPU. Este tipo de situação pode ocorrer quando este processo faz uso intenso
de CPU (é CPU-\emph{bound}) (que faz uso intensivo do processador), o que
geralmente pode ser uma tarefa rotineira de \emph{backup} (compressão,
descompressão ou sincronização de árvores) ou rotinas aritméticas. Neste
caso fica claro que seria melhor ter esta máquina executando no hospedeiro
host1, pois o uso típico de CPU das máquinas virtuais que já estão
executando em host1 ainda pode acomodar vm-11. A dificuldade neste modelo
está no fato de que este uso de CPU pode ser sazonal, e que outras máquinas
virtuais além de vm-11 podem ter situações de demanda abrupta.

Com isso, o software deste projeto atua tentando prever o comportamento
destas máquinas virtuais e requisitando à ferramenta de virtualização que
migre aquelas que apresentarem perspectiva de uso intensivo de CPU para um
hospedeiro que tenha condições de tratar aquela demanda. Este software é
executado na forma de um \emph{daemon}, ou seja, um serviço do sistema
independente aos serviços de virtualização. Este poderá estar instalado em
qualquer uma das máquinas hospedeiras ou até mesmo em uma máquina separada,
providas as credenciais de acesso para a \libvirt{}.

Como dados de entrada, o software recebe as leituras de uso de CPU de cada
uma das máquinas virtuais executando no ambiente. A saída observável do
software é uma série de comandos de migração que são então enviados à
\libvirt{}.

A escolha da ferramenta \libvirt{} como interface de controle do ambiente
de virtualização deve-se ao fato da mesma permitir controlar ambientes
formados for diferentes monitores de máquinas virtuais, como descrito na
seção \ref{sec:libvirt}. A figura \ref{fig:libvirtcontexto0} descreve de maneira
superficial como será a interação entre o software a ser implementado e a
\libvirt{}. O software estará periodicamente coletando informações a respeito
do uso de CPU das máquinas virtuais sendo gerenciadas (1). Com essas
informações, o software fará predições a respeito do comportamento destas
máquinas virtuais e, então, enviará comandos por meio da API (\emph{Application
Programming Interface}, Interface de Programação de Aplicação) da \libvirt{}
(2), comandos estes indicando qual máquina virtual deve migrar para qual
máquina hospedeira. A \libvirt{} por sua vez comunicará a ferramenta de
virtualização (3), que então executará a migração entre um hospedeiro e outro
(4).

\begin{figure}[htp]
\centering
\includegraphics{img-libvirt-contexto0.pdf}
\figinfo{Interação entre \libvirt{} e projeto}
\label{fig:libvirtcontexto0}
\end{figure}

% colocar aqui um ou mais diagramas descrevendo as máquinas virtuais que
% estão em um ambiente "folgado" e então indicar uma situação em que uma
% máquina virtual precisa ser migrada e daí migrar para outra máquina.

\section{Modelagem do software}\label{sec:modelagem}

A figura \ref{fig:contexto0} apresenta o diagrama de contexto,
indicando a relação entre o projeto e a biblioteca \libvirt{}. O fluxo
estatísticas das máq. virtuais indica as informações a respeito de uso de
CPU que são providas pela libVirt, enquanto que cmds. de migração indica os
comandos, transmitidos por meio da API (introduzida na seção
\ref{sec:libvirtapi}), que são enviadas à \libvirt{} para a migração das
máquinas virtuais, de acordo com o que o software a ser implementado
decidir.

\begin{figure}[htp]
\centering
\includegraphics{img-met-contexto-0.pdf}
\figinfo{Diagrama de contexto, interação entre as principais entidades no
sistema}
\label{fig:contexto0}
\end{figure}

Como o projeto será implementado utilizando o conceito de orientação a
objetos, a figura \ref{fig:contexto1} apresenta
de maneira mais detalhada, na forma de um diagrama de contexto, como é a
interação entre os módulos que serão implementados no projeto. O fluxo
“uso de CPU máq. virtuais” (também abreviado como “usos de CPU”)
indica os valores que serão lidos periodicamente de cada máquina virtual.  O
processo Coleta trata da consulta à instância da \libvirt{} e o
pré-processamento dos dados para envio à base. O processo Treinamento
será executado periodicamente e consistirá na execução da fase de treinamento
para as técnicas de AM (Aprendizado de Máquina). O depósito de dados
``base treinada'' representa uma base de dados com os parâmetros
utilizados para representar o conhecimento dos algoritmos de AM. O conteúdo em
si desta base depende da técnica utilizada; SVMs, por exemplo, guardam alguns
exemplos da base de treinamento, já $k$-NN guarda apenas os
exemplos. O processo Predição recebe dois fluxos de dados, o primeiro
é o das informações de treinamento de ``base treinada'', o segundo é
das mesmas informações de CPU que vieram da etapa Coleta; este
processo utiliza o algoritmo de aprendizado para saber se alguma das máquinas
virtuais que estão sendo monitoradas tem possibilidade de saturar os recursos
de processador das máquinas hospedeiras; é esse processo que envia os comandos
necessários à \libvirt{}, visando migrar máquinas para outro hospedeiro que
possa acomodá-las.

\begin{figure}[htp]
\centering
\includegraphics{img-met-contexto-1.pdf}
\figinfo{Diagrama de contexto, interação entre os processos do
projeto proposto}
\label{fig:contexto1}
\end{figure}

% FIXME explicar esse tal pré-processamento

A figura \ref{fig:diagramaclasses0} apresenta um diagrama com as
principais classes do software. A classe Projeto representa uma
instância de todo o projeto e também é responsável pelo controle do ciclo
de vida das classes que fazem coleta, aprendizado e predição. Cada uma
destas classes possui apenas um método público, para que estes sejam
chamados apenas quando a classe Projeto considerar conveniente,
fazendo com que esta classe tenha um papel de escalonador do
software. As classes são a grosso modo um mapeamento dos processos
apresentados na figura \ref{fig:contexto1}, com exceção da classe
InterfaceMMV, que significa Interface para Monitor de
Máquinas Virtuais, a qual tem como propósito centralizar o acesso à \libvirt{},
facilitando a manutenção em caso de mudança de interface, porte para outra
biblioteca ou a simulação de valores (descrita na seção
\ref{sec:desemp}). À exceção de InterfaceMMV, todas estas
classes são abstratas, pois devem ainda ser especializadas para cada uma
das técnicas de aprendizado que serão implementadas no projeto.

\begin{figure}[htp]
\centering
\includegraphics{img-diagrama-classes0.pdf}
\figinfo{Diagrama das principais classes do software proposto}
\label{fig:diagramaclasses0}
\end{figure}

% diagrama de classes

\section{Metodologia de avaliação das técnicas}

Para as etapas de aprendizado e predição, são utilizados os métodos de
aprendizado de máquina \emph{Support Vector Machines} (Máquinas de Vetores
de Suporte) e $k$-NN ($k$-\emph{Nearest Neighbors}). O SVM foi escolhido em
razão de seu bom desempenho em diversos tipos de problemas de aprendizado,
como em detecção de escrita, reconhecimento de faces, categorização de
texto, e outras~\cite{bennett2000support}. $k$-NN foi escolhido por se
tratar de uma técnica, que em alguns casos, pode ter desempenho melhor que
técnicas sofisticadas, servindo como base e indicação de quando uma técnica
pode estar mal implementada, como indicado por
\citeonline{joachims1998text} e \citeonline{hafner2007comparison}.

Assim, das classes que foram apresentadas na seção \ref{sec:modelagem},
serão especializações para cada tipo de técnica de aprendizado. O diagrama
representando estas especializações é apresentado na figura
\ref{fig:diagramaclasses1}.

\begin{figure}[htp]
\centering
\includegraphics{img-diagrama-classes1.pdf}
\figinfo{Diagrama com classes especializado cada técnica de aprendizado
utilizada}
\label{fig:diagramaclasses1}
\end{figure}

% Para o caso de SVMs, será usada a técnica de um-contra-todos (descrita na seção
% \ref{sec:svmmulti}), que consiste na criação múltiplas instâncias de SVM,
% representando a identificação de uma classe.

Para o caso de SVMs, este trabalho seguirá a metodologia proposta
por~\citeonline{hsu2003practical} para a avaliação dos parâmetros que serão
utilizados no sistema. Esta consiste na execução nos seguintes passos:
\begin{enumerate}
  \item Processar os dados coletados para transformar no formato da biblioteca
        que será utilizada (no caso, a \emph{libsvm});
  \item Normalizar os dados (descrita a seguir);
  \item Avaliar o desempenho do uso da função \emph{kernel} RBF 
        ($\exp(-\sigma\norma{\vect{x}_i-\vect{x}_j}^2)$, seção
         \ref{sec:naolinear});
  \item Fazer busca em grade: dos valores do espaço de busca, crescendo os
	 parâmetros da SVM e da função de \emph{kernel} ($C$ e $\sigma$,
         respectivamente) exponencialmente;
  \item Executar testes de validação cruzada, intercalando elementos do conjunto
        de testes com o de treinamento;
  \item Avaliar o desempenho na base de dados completa;
\end{enumerate}

\subsection{Treinamento}

Para a etapa de treinamento, e subsequente uso pela etapa de predição, o
vetor de características usado é constituído de uma janela
formada por um determinado número de leituras anteriores do uso de CPU para
cada máquina virtual controlada naquele domínio. Como estes valores são
percentuais de uso de CPU, a normalização destes é feita com uma divisão
de cada valor por $100$. Na figura \ref{fig:vetorjanela} é apresentada uma
série de leituras de uso de CPU e também como é a configuração da janela
extraída da mesma. Os valores marcados com linhas tracejadas indicam o valor
que é usado como classe utilizada para o treinamento.

\begin{figure}[htp]
\centering
\includegraphics{img-vetor-janela.pdf}
\figinfo{Formação dos vetores de característica a partir da série de leituras de CPU}
\label{fig:vetorjanela}
\end{figure}

Para a classe usada em cada exemplo do treinamento, são utilizados
valores categóricos. Cada classe representa uma faixa de valor do
percentual de uso de CPU. Por exemplo, a faixa 0--24\% seria a classe
A, 25--49\% a classe B, e assim por diante.

\begin{figure}[htp]
\centering
\includegraphics{img-vetor0.pdf}
\figinfo{Exemplo de série de valores lida e composição do vetor de
características}
\label{fig:vetor0}
\end{figure}

Na figura \ref{fig:vetor0}, o item \emph{(a)} apresenta uma janela
formada por valores lidos da série para uma máquina virtual. Em \emph{(b)}
são apresentados os vetores de características que são usados para treinar
(para os testes que usarão SVMs) cada uma das SVMs usadas para categorizar
o uso de CPU. Para este exemplo há 4 classes, o valor $1$ representa a
classe selecionada, e $-1$ a não-seleção dela.

Além disso, são feitos testes com vetores de características contendo
informações a respeito do período do dia, e dia da semana, de maneira a
tentar capturar características sazonais no comportamento das máquinas
virtuais.

\subsection{Predição e migração}

Para a etapa de predição, são utilizados os métodos de classificação das
técnicas de aprendizado de máquina que foram descritas acima. O resultado da
classificação, para cada máquina virtual, irá consistir em uma faixa de
percentual de uso de CPU. Esta faixa então é usada para calcular o
potencial uso total de CPU para o próximo intervalo de tempo (intervalo este
que será diferente durante os testes) e então deve-se decidir se cada máquina
virtual deve permanecer no mesmo hospedeiro ou ser migrada para outra com maior
disponibilidade. Para esta decisão será utilizada a estratégia de \emph{first
fit}, que procura alocar novos objetos no primeiro espaço mais próximo do
começo do espaço disponível~\cite{yao1980new}. Opta-se por utilizar esta
técnica de empacotamento em razão desta ser mais conservadora se comparada a
\emph{first fit decreasing} ou outras técnicas, pois reordenar a lista de
máquinas virtuais pode resultar em desnecessárias migrações entre hospedeiros.

Após o empacotamento das previsões na lista de hospedeiros, aquelas
máquinas virtuais que não estiverem mais em seus hospedeiros originais
são efetivamente migradas utilizando a \libvirt{}.

\section{Avaliação de desempenho}\label{sec:desemp}

% utilizar SVM, K-NN, AR(16) (se for fácil de implementar)
% variar o tamanho das janelas de histórico
% utilizar uma instância do classificador para todas as VMs, versus
% utilizar uma instância nova para cada VM

O processo de validação será constituído de duas etapas. A primeira trata de
avaliar o desempenho de apenas o sistema de predição, utilizando dados reais de
CPU que serão coletados. A segunda etapa irá avaliar o desempenho do sistema
completo em um ambiente utilizando virtualização. Além disso, na primeira etapa
serão ajustados os parâmetros (descritos mais adiante) das técnicas de extração
e treinamento, ao passo que na segunda etapa somente os parâmetros com melhor
desempenho serão utilizados.

Os valores de uso de CPU serão provenientes de um conjunto de máquinas que
fazem parte do ambiente de compilação de software que é utilizado no projeto
\emph{Mandriva Linux}. Tais valores serão lidos por um período de 30 dias, com
o intuito de cobrir as diferentes taxas de uso de CPU que pode-se observar
durante dias úteis e finais de semana.

Este ambiente de compilação é composto por $6$ máquinas que recebem
comandos de compilação de pacotes de software, fazem download códigos fonte
compactados, os descompactam, compilam (que na maioria dos casos são em
linguagem C ou C++) e então os binários e arquivos documentação gerados são
arquivados no formato RPM~\cite{ewing1996rpm}. Estas etapas apresentam
momentos com forte acesso à mídias de armazenamento (\emph{I/O-bound}) bem
como momentos de processamento CPU-\emph{bound}.

\begin{figure}[htp]
\centering
\includegraphics{img-diagrama-classes2.pdf}
\figinfo{Diagrama com classes especializada para simulação e acesso à
\libvirt{}}
\label{fig:diagramaclasses2}
\end{figure}

A leitura destes valores será feita a cada minuto utilizando o utilitário
\emph{top}, que coleta estatísticas de uso de recursos por processos e no
sistema operacional como um todo~\cite{andresen2004monitoring}. Para utilizar
os valores coletados no ambiente de compilação, será criada uma especialização
para a classe InterfaceMMV, como descrito na figura
\ref{fig:diagramaclasses2}, que lerá os arquivos com o histórico de acesso às
máquinas e então entregará estes valores ao módulo de aprendizado.

Para a segunda etapa, o software implementado será testado em um ambiente de
virtualização real, onde cada máquina virtual estará executando pequenos
programas que estarão ocupando as CPUs de maneira a simular os comportamentos
observados no ambiente de compilação descrito anteriormente.  Estes programas
serão implementados utilizando laços de repetição que executarão operações
simples (também chamados de \emph{busy loops}), como implementado também por
\citeonline{dinda2000realistic}.

Finalmente, serão avaliados diferentes valores em parâmetros utilizados
pelo sistema a ser implementado, a saber:

\begin{itemize}
  \item no caso de $k$-NN, o valor de $k$ será alterado, seguindo as
        recomendações de \citeonline{hall2008choice};
   \item o tamanho da janela usada para treinamento e predição, como sugerido
        por \citeonline{frank2001time};
   \item no caso de SVM, o parâmetros $C$, como descrito por \citeonline{ben2010user};
   \item no caso de simulação sem ambiente completo, o número de hospedeiros disponíveis;
   \item a quantidade de faixas dividindo o percentual de uso de CPU para
         uso no atributo categórico da etapa de classificação;
% o que mais?
\end{itemize}

% FIXME não sei que técnicas usar aqui
% A mudança no valor destes parâmetros serão avaliados por meio das seguintes
% métricas:
% \begin{itemize}
%   \item
%   \item bar
% \end{itemize}

Os resultados para cada combinação de parâmetros serão considerados
aceitáveis ou não de acordo com a aderência aos seguintes o
objetivos/questionamentos:
\begin{itemize}
  \item Em momentos de baixo uso de tempo de CPU, foi possível manter as
        máquinas virtuais em um conjunto menor de hospedeiros? (Permitindo
        assim que estas máquinas entrem em estados de menor consumo de
        energia e menor gasto com manutenção.)
  \item Em momentos de alto uso de CPU, as máquinas virtuais mais ativas
        foram migradas para hospedeiros mais ociosos? (Evitando assim a
        degradação do desempenho em outros processos.)
\end{itemize}

\section{Detalhes de implementação}

O projeto é implementado na linguagem \emph{Python}
\cite{rossum1995python}.  Esta possui tipagem dinâmica, suporta o paradigma
de orientação a objetos, facilitando a prototipagem de projetos
\cite{lutz2006programming}.

O ambiente de desenvolvimento será baseado no sistema operacional Linux
\cite{morimoto2004entendendo}, que é a plataforma principal de
desenvolvimento e teste da \libvirt{}. Mesmo a biblioteca sendo acessível
de qualquer sistema operacional, a proximidade com o ambiente de
desenvolvimento diminui a probabilidade de encontrar incompatibilidade
entre ferramentas durante a fase de implementação. A distribuição Linux
utilizada será \emph{Mandriva Linux}\footnote{http://www.mandriva.com/},
que possui em sua base de pacotes todas as ferramentas necessárias para a
implementação. O pacote python-libvirt será utilizado para acessar
os módulos da \libvirt{}.

Para o monitor de máquinas virtuais será utilizado o
KVM,\footnote{http://linux-kvm.org/} que é de código aberto e também
desenvolvido pela empresa Red Hat, garantindo assim maior compatibilidade
com a \libvirt{}.

Para a implementação das instâncias de SVM, será utilizada a
\emph{libsvm}~\cite{chang2001libsvm}. O motivo da escolha desta
implementação deve-se se em razão de sua licença, que permite uso e
distribuição, incluindo uso comercial.
