%
% Metodologia
%
%
% - diagramas de contexto descrevendo quais serão as informações de
%   entrada, qual módulo do software fará o processamento e como serão
%   disponibilizadas as informações da predição,
% - diagrama de classes para a implementação do software,
% 
% - descrição de quais técnicas serão utilizadas e como serão implementadas,
% - descrição de um protocolo para avaliação do desempenho das técnicas
% implementadas.
%
%

\chapter{Metodologia de desenvolvimento}\label{sec:meto}

\section{Introdução}

Em um ambiente de virtualização gerenciado pela ferramenta \libvirt
(apresentada na seção \ref{sec:libvirt}), o software executará em plano de
fundo (ou seja, será um \emph{daemon}) e estará periodicamente coletando
informações quantitativas a respeito do tempo de uso de processador de
cada uma das máquinas virtuais que fazem parte de um determinado
\emph{domínio}. O software utilizará tais informações para treinar um
mecanismo de aprendizado de máquina. Utilizando este mecanismo, a
ferramenta poderá então prever em quais situações alguma máquina virtual
exigirá mais uso de processador que a máquina \emph{host} em que ela
executa terá condições de atender sem prejudicar as outras.

\begin{figure}
\centering
\includegraphics{img-host-guests1.pdf}
\figinfo{Um \emph{host} e várias máquinas virtuais (\emph{guests})}{autoria própria}
\label{fig:hostguests1}
\end{figure}

% colocar aqui um ou mais diagramas descrevendo as máquinas virtuais que
% estão em um ambiente "folgado" e então indicar uma situação em que uma
% máquina virtual precisa ser migrada e daí migrar para outra máquina.

\section{Modelagem do software}

A figura \ref{fig:contexto0} apresenta um diagrama de contexto, de nível 0,
indicando a relação entre o projeto que será desenvolvido e a biblioteca
\libvirt. O fluxo \texttt{estatísticas das máq. virtuais} indica as
informações a respeito de uso de CPU que são providas pela libVirt,
enquanto que \texttt{cmds. de migração} indica os comandos, transmitidos
por meio da API (descrita na seção \ref{sec:libvirtapi}), que são enviadas
à \libvirt{} para a migração das máquinas virtuais, de acordo com o que o
software a ser implementado decidir.

\begin{figure}
\centering
\includegraphics{img-met-contexto-0.pdf}
\figinfo{Diagrama de contexto, interação entre as principais entidades no
sistema}{autoria própria}
\label{fig:contexto0}
\end{figure}

% diagrama de classes

\section{Implementação}


\section{Treinamento}

\section{Avaliação de desempenho}\label{sec:desemp}

% utilizar SVM, K-NN, AR(16) (se for fácil de implementar)
% variar o tamanho das janelas de histórico
% utilizar uma instância do classificador para todas as VMs, versus
% utilizar uma instância nova para cada VM
