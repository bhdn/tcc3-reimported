%% vim:ft=tex
\documentclass[font=plain,chapter=TITLE,section=Title,espaco=duplo,tocpage=plain,appendix=Name,floatnumber=continuous]{abnt}
\usepackage{hyperref}
\usepackage[utf8]{inputenc}
\usepackage[brazil]{babel}
\usepackage[alf]{abntcite}
\usepackage{graphicx}
\usepackage{timestamp}
\usepackage{amssymb,amsmath}

%% xunxo para seguir as normas da UTP
\usepackage{xunxos-utp}

%% informações sobre o trabalho
\autor{Bogdano Arendartchuk}
\titulo{Avaliação de técnicas de predição de comportamento de máquinas
virtuais para a otimização de uso de parque computacional}
\comentario{Trabalho de Conclusão de Curso apresentado ao Curso de Bacharelado
em Ciência da Computação, da Faculdade de Ciências Exatas da Universidade
Tuiuti do Paraná, como requisito parcial para a obtenção do grau de Bacharel em
Ciência da Computação.}
\instituicao{Universidade Tuiuti do Paraná}
\orientador[Orientador: ]{Islenho Almeida}
\local{Curitiba}
\data{2010}

\begin{document}
\renewcommand{\figurename}{FIGURA}
\citeoption{minhasopcoes}
\nohyphens


\UTPCapa
\UTPFalsaFolhaDeRosto
\UTPFolhaDeRosto

\begin{resumo}
% FIXME Neusa fez a seguinte nota:
% "O resumo deve dar noção do que o leitor encontrará pela frente em termos
% de conteúdo e não em termos cronológicos de assuntos."
Este trabalho apresenta uma metodologia de um sistema de predição de
comportamento de máquinas virtuais em um ambiente gerenciado pela
ferramenta libVirt. São estudadas as técnicas de aprendizado Máquinas de
Vetores de Suporte e $k$-Vizinhos mais Próximos, além dos conceitos e
ferramentas relacionados a virtualização. Por fim são apresentados os
modelos, processos e técnicas que serão utilizados no sistema.

Palavras-chave: virtualização; aprendizado de máquina; libVirt; Linux;
máquinas de vetores de suporte.
\end{resumo}

\listoffigures
%\listoftables
\listadequadros
\sumario

% definições globais para o trabalho

\newcommand{\libvirt}{\emph{libVirt}}



%
% Introdução
%

\chapter{Introdução}

Esta é a introdução.


%
% Revisão da Literatura
%

\chapter{Revisão da literatura}

\section{Aprendizado de Máquina}\label{sec:aprendizado}

% 
% Aprendizado de máquina
%

\newcommand{\vect}[1]{\mathbf{#1}}
\newcommand{\norma}[1]{\lVert #1 \rVert}

% Usar Inductive Principles for Learning from Data como referência "dos
% tópicos" para esse assunto, e também o "Uma Introdução às Support Vector
% Machines".
% Tomar cuidado em manter a consistência entre as diferentes notações
% usadas entre as referências.
%
% - A teoria do aprendizado estatístico
% - Métodos paramétricos
% - Métodos adaptativos
% - Risco esperado, risco estatístico
% - Classificação binária
% - Classificação multi-classes
% - Como adaptar um classificador binário para um problema multi-classes
%   (Pairwise, e esqueciooutro)
%
% Apresentar (brevemente) algumas outras técnicas também:
% (lista baseada em wiki/Supervised_learning#Approaches_and_algorithms)
% - redes neurais artificiais
% - inferência bayesiana
% - ...
%
% FIXME Aqui realmente estou em dúvida. Devemos apresentar TODAS as
% técnicas, uma por parágrafo, ou apresentar uma a uma?
% -- Pelo que tenho visto em outras monografias, não é necessário
%  apresentar o estado da arte da área de aprendizado (Ver Trabalhos
%  Relacionados no Google Docs)

Um dos comportamentos mais característicos de sistemas computacionais que possam ser considerados inteligentes é o de melhorar o seu desempenho para resolver algum problema quando este se depara com a mesma situação por mais de uma vez ~\cite{luger2004inteligencia}. ~\apudonline{simon1983should}{luger2004inteligencia} define o conceito de aprendizado de máquina como sendo qualquer mudança que melhore o seu desempenho na segunda vez que ele repetir a mesma tarefa, ou numa tarefa da mesma população.

Nem sempre é possível obter um resultado ótimo para novas situações, pois na maioria dos casos não é possível apresentar a um sistema a quantidade de exemplos (ou experiências anteriores) necessárias para representar todos os casos possíveis. Por isso, as técnicas de aprendizado fazem uso da inferência indutiva, que consiste em prever novas situações baseando-se em um conjunto particular de exemplos~\cite{luger2004inteligencia}.

A construção da hipótese no aprendizado indutivo ocorre por meio da exposição de exemplos ao algoritmo de indução, e este processo é chamado de treinamento. O processo de atribuir rótulos ou classes a novos exemplos é chamado de classificação.

Porém, se o conjunto de exemplos usado para a indução for muito pequeno ou pouco representativo, há o risco de que as hipóteses escolhidas não descrevam corretamente novas situações e o sistema inteligente traga resultados incorretos~\cite[p. 90]{rezende2003sistemas}.

Sistemas de aprendizado de máquina podem ser divididos em dois tipos de acordo com a maneira que é feito o treinamento: supervisionados e não-supervisionados.

O aprendizado supervisionado é aquele em que há exemplos pré-existentes e estes já possuem rótulos. Estes rótulos podem ter sido atribuídos por um especialista no domínio ou representam resultados de experiências anteriores do sistema. Um classificador baseado em aprendizado supervisionado deve tentar estabelecer uma hipótese que seja genérica a ponto de que o algoritmo possa classificar corretamente novos casos. Uma hipótese muito genérica pode resultar em uma classificação inadequada (\emph{underfitting} --- sub-ajuste), ao passo que uma hipótese que descreva corretamente apenas os exemplos que foram usados para o treinamento pode ter um desempenho inadequado em casos novos (\emph{overfitting} --- sobre-ajuste).

No aprendizado não-supervisionado, não há rótulos para atribuir aos exemplos, pois o objetivo do algoritmo é encontrar grupos de exemplos que possuam semelhança entre eles. As técnicas deste modelo são referenciadas como sendo ``técnicas de agrupamento'' ou \emph{clustering}~\cite{rezende2003sistemas}.
%FIXME terminar não-supervisionado (dar uma olhada no livro do Luger)

\subsection{Técnicas de aprendizado supervisionado}

Dentre as técnicas de aprendizado supervisionado mais populares estão Redes Neurais, Máquinas de Vetores de Suporte (\emph{Support Vector Machines} --- SVMs), $k$ Vizinhos Mais Próximos ($k$-NN) e \emph{Naïve Bayes}.

% FIXME faltou descrever GGM, RBF, Redes Bayesianas e árvores de decisão

$k$-NN (de \emph{k-Nearest Neighbors}, $k$-Vizinhos Mais Próximos) é uma técnica de aprendizado supervisionado que utiliza todos os exemplos durante a etapa de classificação, tendo assim pouco custo para treinamento e alto custo para classificação. Para classificar as entradas, comumente mede-se a distância euclideana de todos os exemplos da base e então é selecionada a classe que predomina no grupo de $N$ elementos que estiver mais próximo\cite{tan2006effective}.

Há também as técnicas baseadas em Árvores de Decisão, que visam a indução de uma árvore que represente o conhecimento. Os ramos desta árvore indicam conjunções (tanto de testes binários quanto multivalorados), ao passo que as folhas representam classficações. Para classificar novos casos simplemente percorre-se a árvore de decisão avaliando cada conjunção em cada ramo e seguindo para a subárvore adequada, quando atinge-se as folhas é encontrada a classe final~\cite{de-categorizacao}. Esta técnica possui a vantagem de que sua forma de representação permita que humanos compreendam com facilidade o conteúdo que ela representa~\cite[p. 52]{mitchell1997machine}. Os algoritmos para indução de árvores de decisão mais populares são ID3~\cite{quinlan1986induction}, C4.5~\cite{quinlan1993programs} e CART~\cite{breiman1984classification}.

\emph{Naïve Bayes} é um algoritmo de aprendizado supervisionado, que
utiliza probabilidade para a classificação e supõe que as variáveis de cada
exemplo são condicionalmente independentes \cite{de-categorizacao}, as
probabilidades são calculadas de acordo com o Teorema de Bayes
\cite{kim2003poisson}, para a classificação de um elemento desconhecido, é
calculada a probabilidade de todas as classes e a classe com maior
probabilidade é escolhida como rótulo para o elemento desconhecido.
~\cite{pardo2002aprendizado}

As Redes Neurais Artificiais (RNAs) são modelos matemáticos baseados no neurônio biológico, têm a capacidade computacional adquirida por meio de aprendizado e generalização, são capazes de resolver problemas de aproximação, predição, classificação e otimização. A aprendizagem de uma rede neural ocorre a partir dos exemplos e tende a melhorar seu desempenho conforme a qualidade dos exemplos, já a generalização surge como consequência do aprendizado. O processamento das informações pode ser feito de forma paralela e distribuída, através dos neurônios da rede neural artificial, onde cada neurônio é um elemento processador.~\cite{rezende2003sistemas}

\emph{Support Vector Machines} é uma técnica modelada como um problema de otimização que tenta encontrar, no espaço do conjunto de exemplos, um hiperplano que separe com maior margem os exemplos de cada uma das duas classes. Em razão da aplicação desta técnica em parte da metodologia proposta neste trabalho, será feita uma apresentação mais detalhada a partir da seção \ref{sec:svm}) (página \pageref{sec:svm}). Antes disso, as seções a seguir buscam apresentar os conceitos e terminologia comumente utilizados na apresentação das SVMs.

\subsection{Formalização do aprendizado supervisionado}

Como já brevemente apresentado, no problema de aprendizado supervisionado existe a figura de um professor que indica qual é o rótulo correto para cada exemplo \apud{haykin1994neural}{lorena2003introducaoas}. Sendo assim, cada exemplo pode ser descrito na forma $(\vect{x}_i, y_i)$, aonde $\vect{x}_i$ denota um exemplo e $y_i$ seria a classe ou rótulo. Também, após o processo de treinamento, pode-se descrever o classificador como uma função $f(\vect{x}) = y$, sendo que $\vect{x}$ não é necessariamente um dos valores de $\vect{x}_i$.

Os valores dos rótulos que os exemplos podem assumir podem ser discretos ou contínuos. Para o caso dos contínuos assume-se que é possível obter $1,\dotsc,k$ valores. Quando $k = 2$, o problema é denominado como sendo de ``classificação binária''. Quando $k > 2$ denomina-se como sendo um problemas de ``classificação multiclasses''.

Os exemplos $\vect{x}_i$, são representados por vetores com as características (também denominadas atributos) de cada exemplo. Cada exemplo $\vect{x}_i$ possui $m$ atributos e também pode ser representado como $\vect{x}_i = (x_{i1},\dotsc,x_{im})$. Os atributos podem assumir dois tipos de valores: nominais ou contínuos. Os atributos nominais assumem valores que não possuem ordem entre si e sua representação tem função simbólica (por exemplo: segunda-feira, azul, não). Os atributos contínuos possuem ordem entre si e comumente representam valores dos domínios $\mathbb{Z}$ e $\mathbb{R}$.

O objetivo de uma técnica de aprendizado de máquina é obter uma função $f(\vect{x}) = y$ que obtenha um $y$ adequado aos exemplos que foram apresentados pelo professor por meio de indução~\cite{osuna1997support}.

% parece ser muito pouco conteúdo, porém mais detalhes como (overfitting) ou
% (como avaliar o desempenho de alguma técnica de AM) já foram (acima) ou serão
% (em classificação de textos).

\subsection{Support Vector Machines --- \it{SVMs}}\label{sec:svm}
%
%   - problema primal
%   - problema dual (e teoria "dos lagrangianos")
%   - vetores de suporte
%   - SVM com margens rígidas
%   - SVM com margens flexíveis (E_i)
%   - kernels (RBF gaussiano etc)

A técnica de aprendizado de máquina supervisionado conhecida como \emph{Support Vector Machine} (SVM) ou Máquinas de Vetores de Suporte, foi introduzida por Boser, Guyon e Vapnik em 1992 com a publica\c{c}ão de \emph{A training algorithm for optimal margin classifiers}\nocite{boser1992training}. Esta baseia-se no trabalho da teoria do aprendizado estatístico desenvolvida por Vapnik et al. desde a década de 1960~\cite{antos2003data}.

SVMs usam o princípio de que quanto mais largas as margens de um hiperplano separador de uma fun\c{c}ão, que foi explicado na se\c{c}ão \ref{sec:aprendizado} por meio do conceito de minimiza\c{c}ão do risco estrutural, maiores são as chances de que ele possa prover uma boa generaliza\c{c}ão dos dados que estão sendo usados como exemplo \cite{chapelle2002choosing}. Então, SVMs buscam encontrar um hiperplano descrito por $f(\vect{x}) = \vect{w}\cdot\vect{x} + b = 0$ que tenha maior margem entre as classes e menor complexidade estrutural por meio da resolu\c{c}ão de um problema de otimiza\c{c}ão.

Para encontrar o hiperplano separador ideal, é necessário utilizar (pelo menos) dois pontos do conjunto de exemplos. Sejam $\vect{x}_1$ e $\vect{x}_2$ dois exemplos do conjunto de treinamento $T$ que possui um conjunto de exemplos $X$ com rótulos do conjunto de exemplos $Y = \{-1, +1\}$, e cada um deles fica em um lado diferente do hiperplano separador. Para encontrar o hiperplano separador $f(\vect{x}) = \vect{w}\cdot\vect{x} + b = 0$ entre $\vect{x}_1$ e $\vect{x}_2$, é necessário conhecer o vetor $\vect{w}$, que deve ser normal a este hiperplano e que é usado para calcular o tamanho da margem.

\begin{figure}[h!]
  \centering
  \includegraphics[width=0.5\textwidth]{img/fig-hiperplanos.png}
  \figinfox{Hiperplano separador e margens}{LORENA, 2003}
  \label{fig:hiperplanos}
\end{figure}

Como um dos objetivos do problema de otimização é obter uma margem larga entre os exemplos, pode-se restringir a busca por $\vect{w}$ apenas em termos do cálculo do tamanho da margem. Para isso, é necessário calcular a distância entre dois hiperplanos que são formados pelos exemplos $\vect{x}_1$ e $\vect{x}_2$. Sejam $H_1: \vect{w}\cdot\vect{x} + b = +1$ e $H2: \vect{w}\cdot\vect{x} +b = -1$ dois hiperplanos que ficam paralelamente acima e abaixo, respectivamente, do hiperplano separador. Além disso, assume-se que $H_1$ passa por $\vect{x}_1$ e $H_2$ passa por $\vect{x}_2$. A figura \ref{fig:hiperplanos} mostra a relação entre os hiperplanos e os pontos $\vect{x}_1$ e $\vect{x}_2$.

Conhecendo os hiperplanos, agora é possível calcular a distância entre os hiperplanos que servem de fronteira entre cada ponta da margem do hiperplano separador. A equação \ref{eq:projecao_w} apresenta o cálculo necessário para projetar $\vect{x}_1 - \vect{x}_2$ na direção de $\vect{w}$, que é perpendicular ao hiperplano separador $\vect{w}\cdot\vect{x} + b = 0$.
\begin{equation}\label{eq:projecao_w}
  (x_1 - x_2)
    \left(
      \frac{ \vect{w} }{ \norma{\vect{w}} }
      \cdot
      \frac{(x_1 - x_2)}{ \norma{\vect{x}_1 - \vect{x}_2} }
    \right)
\end{equation}

Sabendo que $\vect{w}\cdot\vect{x}_1 + b = +1$ e $\vect{w}\cdot\vect{x}_2 + b = -1$, e levando em conta que deseja-se saber o comprimento do vetor resultante, é usada a norma da equação \ref{eq:projecao_w} para chegar à equação \ref{eq:projecao_w2}, que indica a distância $d$ utilizada na figura \ref{fig:hiperplanos}:
\begin{equation}\label{eq:projecao_w2}
  d = \frac{2}{\norma{\vect{w}}}
\end{equation}

Portanto, as distâncias entre $\vect{x}_1$, $\vect{x}_2$ e o hiperplano separador $\vect{w}\cdot\vect{x} + b = 0$ são $\frac{1}{\norma{\vect{w}}}$, e com isso, é possível definir o problema de otimização como definido pelo problema de otimização \ref{eq:max_w0}. A restrição \ref{eq:max_w1} indica que $H_1$ e $H_2$ devem passar, respectivamente, pelos vetores $\vect{x}_1$ e $\vect{x}_2$.
\begin{eqnarray}
& \label{eq:max_w0}\operatorname{Maximizar} & \frac{2}{\norma{\vect{w}}} \\
& \label{eq:max_w1} \operatorname{sujeito\;a} & y_i(\vect{w}\cdot\vect{x}_i + b) - 1 \ge 0 \quad i = 1,\dotsc,n.
\end{eqnarray}

No problema de maximização \ref{eq:max_w0}, $\frac{2}{\norma{\vect{w}}}$ pode também ser descrito como um problema de minimização de $\norma{\vect{w}}^2/{2}$:
\begin{eqnarray}
& \label{eq:min_w0}\operatorname{Minimizar} & \frac{\norma{\vect{w}}^2}{2} \\
& \nonumber \operatorname{sujeito\;a} & y_i(\vect{w}\cdot\vect{x}_i + b) - 1 \ge 0 \quad i = 1,\dotsc,n.
\end{eqnarray}

A partir deste ponto, o problema \ref{eq:min_w0} pode ser resolvido com técnicas de programação quadrática (PQ) \cite{osuna1997support}. Este tipo de problema pode ser solucionado utilizando uma função Lagrangiana e adicionando as restrições à função objetivo junto com os multiplicadores de Lagrange $\alpha_i$ \cite{smola2000advances}. A equação \ref{eq:lagrange0} deve ser minimizada, o que significa maximizar $\alpha_i$ e minimizar $\vect{w}$ e $b$. O problema é representado desta forma para que a restrição \ref{eq:max_w1} possa ser representada na forma dos multiplicadores $\alpha_i$, o que facilita os cálculos mais adiante, e também porque os dados de treinamento apenas aparecem na forma de produtos entre vetores \cite{burges1998tutorial}, o que permite o uso de \emph{kernels}, que são apresentados na seção \ref{sec:naolinear}.
\begin{equation}\label{eq:lagrange0}
  L(\vect{w}, b, \vect{\alpha}) = \frac{1}{2}\norma{\vect{w}}^2 - 
       \sum_{i=1}^n{\alpha_i(y_i(\vect{w}\cdot\vect{x}_i + b) - 1)}
\end{equation}

% TODO REV (pg16) "escrever um parágrafo definindo o polinômio de lagrange"

Tem-se ponto de sela, então:
\begin{eqnarray}\label{eq:sela}
  \frac{\partial{L}}{\partial{b}} = 0 & \text{e} & \frac{\partial{L}}{\partial{\vect{w}}} = 0
\end{eqnarray}
E com isso:
\begin{eqnarray}
&  \label{eq:lagsum0}  \sum_{i=1}^n{\alpha_i y_i} = 0 \\
&  \label{eq:lagsum1}  \vect{w} = \sum_{i=1}^n{\alpha_i y_i\vect{x}_i}
\end{eqnarray}

Assim, substituindo equações \ref{eq:lagsum0} e \ref{eq:lagsum1}, é possível formular o problema de otimização:
\begin{eqnarray}\label{eq:maxalpha}
\begin{array}{rl}
   %\operatorname*{Maximizar}_{\alpha}
   \underset{\alpha}{\operatorname{Maximizar}}
   & \sum_{i=1}^n{\alpha_i - \frac{1}{2}}
                            \sum_{i,j=1}^n{\alpha_i\alpha_j y_i y_j(\vect{x}_i\cdot\vect{x}_j)} \\
\operatorname{sujeito\;a} &
  \begin{cases}
    \alpha_i \geqslant 0, \forall{i} = 1,\dotsc,n \\
    \sum_{i=1}^n{\alpha_i y_i} = 0
  \end{cases}
\end{array}
\end{eqnarray}

A equação \ref{eq:maxalpha} é denominada a forma dual do problema, ao passo que a formulação original na equação \ref{eq:min_w0} é denominada a forma primal, baseada no trabalho de \apudonline{fletcher1987practical}{burges1998tutorial}.

% TODO REV Marcio pg16, "o que é forma primal e forma dual?"
% TODO REV Marcio pg16, "quais são as condições [de KKT]"

É possível utilizar as condições KKT (de Karush-Kuhn-Tucker), descritas em \citeonline[proposição 3.3.1]{bertsekas-nonlinear}, visto que o problema de otimização \ref{eq:maxalpha} possui restrições lineares e a função objetivo é convexa \cite{burges1998tutorial}. Assim, segundo essas condições, é possível encontrar $\vect{w^*}$ e $b^*$ que podem ser considerados solução ótima para o problema a partir da solução do problema dual ao encontrar $\alpha_i^*$:
\begin{equation}\label{eq:dualalpha}
  \alpha_i^*(y_i(\vect{w^*}\cdot\vect{x}_i+b^*) - 1) = 0, \forall{i}=1,\dotsc,n
\end{equation}

Nesta equação $\alpha_i^*$ é diferente de zero apenas para os valores que tocam a borda das margens do hiperplano de decisão ($H_1$ e $H_2$). Assim, esses dados são chamados de ``vetores de suporte'', pois são os dados mais significativos para a localização de hiperplano $\vect{w}\cdot\vect{x} + b = 0$.

E para calcular $b^*$, de acordo com \ref{eq:dualalpha}:
\begin{equation}\nonumber
  b^* = \frac{1}{n_{SV}}\sum_{x_j \in SV}{\frac{1}{y_j} - \vect{w^*}\cdot\vect{x}_j}
\end{equation}

Aonde $n_{SV}$ é o número de vetores de suporte e $SV$ o conjunto dos mesmos.

% ALELUIA!!!!!!!!!!!! PQPQPQPQPQPQPQP!!!
Finalmente, obtém-se a seguinte função classificadora:
\begin{equation}\label{eq:classificadora}
  g(\vect{x}) = \operatorname{sinal}(f(\vect{x}))
              = \operatorname{sinal}\left(
                  \sum_{x_i \in SV}{y_i\alpha_i^*\vect{x}_i\cdot\vect{x}+b^*}
                \right)
\end{equation}

\subsubsection{Margens suaves}

% seção está incompleta, não explica bem como o erro é tratado em si

Para tratar os casos em que há \emph{outliers} nos exemplos, ou seja, dados que estão rotulados incorretamente ou com algum ruído, utiliza-se a técnica de margens suaves, que é uma saída mais simples que SVMs não-lineares \cite{burges1998tutorial}.

Utiliza-se variáveis de folga $\xi_i$ para cada exemplo $\vect{x}_i$ do conjunto de treinamento. Essas variáveis são adicionadas à restrição do problema primal:
\begin{equation}\label{eq:restrsuave}
  y_i(\vect{w}\cdot\vect{x}_i+b) \geqslant 1 - \xi_i,\quad\forall{i}=1,\dotsc,n
\end{equation}

Ao passo que a função objetivo é reformulada como:
\begin{equation}
  \underset{\vect{w}, b, \vect{\xi}}{\operatorname{Minimizar}}\quad
         \frac{1}{2}\norma{\vect{w}}^2+C\left(\sum_{i=1}^n{\xi_i}\right)
\end{equation}

A constante $C$ impõe uma penalização à violação das restrições \ref{eq:restrsuave} do problema de otimização. O valor desta constante é definida pelo usuário e sua definição depende de testes baseados no conjunto de treinamento. Algumas abordagens para a escolha deste parâmetro foram apresentadas por \citeonline{chapelle2002choosing}, \citeonline{cherkassky2004practical}, \citeonline{quang2002evolving} e \cite{ben2010user}.

% FIXME descrever o método de busca em grade de ben2010user, pois é o que
% usamos

\subsubsection{Classificação não-linear}\label{sec:naolinear}

Segundo o teorema de \citeonline{cover1965geometrical}, as chances de que um conjunto de exemplos não linearmente separável possa ser separado por um hiperplano é grande quando este é disposto em um espaço de maior dimensionalidade. Assim, a implementação das máquinas de vetores de suporte utiliza esta técnica para conseguir separar dados ainda de maneira linear~\cite{burges1998tutorial}.

Com isso, os dois vetores $\vect{x}$ e $\vect{x}_i$, que são utilizados na função de decisão \ref{eq:classificadora}, são convertidos para o espaço de maior dimensão por um mapeamento $\vect{\Phi}: X \rightarrow \Im$, aonde $X$ é o espaço de entrada e $\Im$ o espaço de características.

Adicionalmente, o produto entre os vetores $\vect{x}$ e $\vect{x}_i$ é representado como uma função $K(\vect{x}_i, \vect{x}_j) = \vect{\Phi}(\vect{x}_i)\cdot\vect{\Phi}(\vect{x}_j)$. Esta função é chamada de \emph{função kernel} e tem por finalidade permitir o uso de funções que atentem às condições definidas pelo teorema de Mercer \cite[p. 141]{burges1998tutorial}. Esta forma de representação permite que se use estas funções de kernel dentro da implementação de uma SVM sem a necessidade de conhecimento dos detalhes internos das mesmas. O quadro \ref{quadro:kernels} apresenta alguns dos kernels mais populares utilizados com SVMs~\cite{lorena2003introducaoas}.

\begin{table}
\centering
\begin{tabular}{|c|c|c|}
\hline
Tipo de kernel & Função & Parâmetros \\
\hline
Polinomial & $(\delta(\vect{x}_i\cdot\vect{x}_j)+\kappa)^d$ & $\delta$, $\kappa$, e $d$ \\
Gaussiano (ou RBF) & $\exp(-\sigma\norma{\vect{x}_i-\vect{x}_j}^2)$ & $\sigma$ \\
Sigmoidal & $\tanh(\delta(\vect{x}_i\cdot\vect{x}_j)+\kappa)$ & $\delta$ e $\kappa$ \\
\hline
\end{tabular}
\quadro{Funções de kernel comumente usadas}\label{quadro:kernels}
\end{table}

\subsubsection{Classificação multiclasses}\label{sec:svmmulti}

% hsu2002comparison

Como SVMs fazem classificação binária, é necessário o uso de alguma técnica para adaptar problemas de classificação multiclasses. As mais populares são um contra todos (\emph{one-against-all}) e um contra um (\emph{one-against-one}).

Na técnica um contra todos, é treinada uma SVM para cada classe contra todas as outra classes ao mesmo tempo. \citeonline{vapnik1998statistical} propôs uma extensão a esta técnica para utilizar os valores contínuos de cada SVM (ao invés do retornado por $\operatorname{sinal}$) e ordenar as classes descendentemente de acordo com o módulo da classificação de cada uma~\cite{abe2003analysis}.

Na técnica de classificação um contra um, ou também chamado de \emph{pairwise}, cada classe é treinada contra outra classe do problema; a classe selecionada é a que foi selecionada mais vezes nas classificações contra todas as outras classes. Isso resulta, em um problema de classificação de $n$ classes, em $n(n - 1)/2$ SVMs \cite{kressel1999pairwise}.

% TODO há mais técnicas interessantes (as baseadas em árvore de decisão e
% all-together), talvez apontar desempenho delas etc etc

%%\subsubsubsection{Complexidade computacional}
%% tratar

%%\subsubsubsection{Implementações}
% abandonada por falta de tempo, vamos citar apenas a libsvm na metodologia
% libsvm 
% PyML
% svm-light
% svm-perf



\section{Virtualização}\label{sec:virt}

O conceito clássico de virtualização consiste na execução simulada de cada
instrução de uma determinada arquitetura bem como a emulação de seu
hardware original, de maneira a permitir que qualquer sistema operacional
compatível possa executar também neste ambiente~\cite{goldberg1974survey}.
Diferentemente do que se observa na arquitetura clássica de computadores,
aonde cada componente provê uma abstração os componentes mais próximos do
usuário, a virtualização multiplexa os serviços de hardware para
componentes virtuais, que possam ser utilizandos por sistemas operacionais
sem adaptações XXX. % FIXME citar os br da 'revisao de virtualizacao'

Atribui-se ao sistema \emph{OS/360} a introdução do conceito de
virtualização, que neste caso era limitada a apenas uma instância de uma
\emph{máquina virtual} para os processos não-privilegiados. Com o passar do
tempo isto foi aprimorado com o desenvolvimento dos \emph{Monitores de
Máquinas Virtuais} (também conhecidos como hipervisores ou
\emph{hypervisors}), que eram softwares especializados em executar um
conjunto de instruções de alguma arquitetura e que permitiam então, por
estar não relacionado a um serviço de sistema operacional, executar
múltiplas máquinas virtuais, permitindo assim que o custo computacional com
a execução simulada das instruções fosse compensado com a execução de
muitas máquinas virtuais, aproveitando assim o tempo de
CPU~\cite{goldberg1974survey}.

Com a popularização da internet e o barateamento do hardware, a
virtualização começou a ser utilizada com o intuito de facilitar a
manutenção em ambiente de servidores. A facilidade da criação e
configuração permitiu que serviços executem em máquinas (virtuais)
dedicadas, aumentando a segurança (por meio do isolamento) e facilitando a
manutenção com menor risco de comprometimento de outros
serviços~\cite{smith2005architecture}.

% FIXME citar aqui introdução do conceito de cloudcomputing

% FIXME começar a descrever tolerância a faltas, e então descrever migração
% e as tais técnicas de sincronização de blabla

% FIXME remover essa porcaria
É importante apresentar alguns conceitos que são amplamente utilizados ao
tratar de virtualização, como apresentado por
\citeonline{peter2005resource}:
\begin{description}
  \item[hóspede] refere-se à máquina virtual;
  \item[hospedeiro] refere-se ao Monitor de Máquinas Virtuais;
  \item[paravirtualização] conjunto de técnicas no qual o hóspede oferece
        serviços ao hospedeiro, que por sua vez possui suporte explícito a
        eles;
  \item[migração] ocorre quando todo o estado de um hóspede
       (memória, registradores, dispositivos) que está executando em um
       determinado hospedeiro é transfeirido para outro, de maneira que a
       execução do hóspede prossegue, sem que software executando dentro
       desta note diferença.
\end{description}

\subsection{Ferramentas}

Atualmente existe uma grande diversidade de ferramentas que atuam como
monitores de máquinas virtuais. Algumas utilizam implementações de
virtualização utilizando técnicas mais simples e com maior custo
computacional, ao passo que outras são mais sofisticadas, além do uso de
recursos de virtualização disponíveis em CPUs modernas.

A família de produtos oferecida pela empresa VMWare está entre as mais
populares ferramentas de virtualização utilizadas atualmente.

Além disso, uma outra ferramenta muito popular para virtualização é a
VirtualBox.

% mais ferramentas interessantes: bochs, denali, ibm rhype, virtualbox

% introdução
% ferramentas

% descrição de medidas de carga de CPU: loadavg e percentual de uso de CPU

\subsection{\libvirt}\label{sec:libvirt}

% uma descrição mais detalhada, 
% virt-manager
% virsh
% "conceitos"

A \libvirt{} é um conjunto de ferramentas que visa prover uma abstração
uniforme para gerenciamento de máquinas virtuais de maneira independente da
da aplicação de virtualização que está sendo utilizada. Seu desenvolvimento
foi iniciado em 2005 pela empresa Red Hat e inicialmente chamava-se
\emph{libxen} (indicando assim que era uma abstração específica para a
ferramenta \emph{Xen}), porém teve seu nome alterado para \libvirt{} em
2006, de maneira a refletir o interesse em suportar mais ferramentas de
virtualização \apud{gitlibvirt}{eriksson2009virtualization}.

\subsubsection{Interface de programação}\label{sec:libvirtapi}

\emph{Trecho ainda não concluído}.


%
% Trabalhos relacionados
%

\chapter{Trabalhos relacionados}

Pode-se observar que a área que possui uma grande quantidade de artigos
relacionados ao problema de tentar prever o uso de CPU é a de sistemas
distribuídos. Isso se deve ao fato de que em certos tipos de sistemas
distribuídos, é necessário que tarefas sejam despachadas para execução por
unidades de processamento que, preferencialmente, estejam ociosas~\cite{zhang2007cpu}.

Os trabalhos de Dinda (2000)\nocite{dinda2000host} e de Dinda
(2002)\nocite{dinda2002evaluation} efetuam a coleta de informações de uso de
CPU em um conjunto formado por 38 computadores de diferentes tipos que
pertenciam ao \emph{Carnegie Mellon University} e também no \emph{Pittsburgh
Supercomputing Center}, nos Estados Unidos, pelo período de pouco mais que uma
semana. Com estas informações de uso de CPU foi feita uma análise deste conjunto
de dados e avaliou-se algumas técnicas de modelagem de séries temporais, e
finalmente sugere que seja usado o modelo AR(16) (\emph{Autoregressive model},
modelo auto-regressivo, com 16 coeficientes) para a predição de comportamento
de CPUs. Este modelo consiste em uma representação na forma
$x_t = \phi_{1}x_{t-1} + \phi_{2}x_{t-2}+\dotsc,\phi_{p}x_{t-p}+e_t$, aonde a
$\phi_p$ correspondem aos parâmetros do modelo (sendo $p$ a ordem do mesmo) e
$e_t$ representa um ruído no modelo. Utilizando a técnica de Yule-Walker é
possível conhecer os parâmetros $\phi_p$ do modelo e assim prever itens na
série calculando-se $x_{t+1}$~\cite{baddour2005autoregressive}.

O trabalho de~\citeonline{sodhi2008performance} apresenta uma técnica de
predição de comportamento de processos baseando-se na execução de uma pequena
aplicação que tenta representar aspectos significativos de uma aplicação que
tem alto custo computacional de execução. A ferramenta apresentada coleta
informações a respeito do software que está sendo analisado e então geral um
“esqueleto” da mesma, que então será executado em um ambiente que será
avaliado. Para avaliação de desempenho foram usadas ferramentas do conjunto NAS
Benchmark, com esqueletos de tempo de execução variando de 0,5 a 10 segundos.
Com exceção do esqueleto de 0,5 segundos, os outros tiveram uma taxa de erro de
predição média de 6,7\%. 

\citeonline{wolski2000predicting} compara duas medidas amplamente disponíveis
em sistemas Unix para avaliação de uso de CPU, a \emph{vmstat} que provê
informações de tempo gasto com processamento dentro de \emph{kernel} nos
aplicativos de usuário e a medida \emph{load average}, que é a contagem do
número de processos que estavam na fila de execução do escalonador no último
minuto. Foram feitas predições utilizando o \emph{Network Weather Service}
\cite{wolski1999network}, que é um conjunto de ferramentas distribuído para a
previsão de séries temporais, para a avaliação dessas medidas. As avaliações
demonstraram que ambas tem baixo erro para previsões de curto (10 segundos) e
médio (5 minutos) prazo.

% tese do eugeni
% dinda (análise de comportamento dos hosts, virtuoso)
% skeletons
% homeostatic,
% etc


%
% Metodologia
%
%
% - diagramas de contexto descrevendo quais serão as informações de
%   entrada, qual módulo do software fará o processamento e como serão
%   disponibilizadas as informações da predição,
% - diagrama de classes para a implementação do software,
% 
% - descrição de quais técnicas serão utilizadas e como serão implementadas,
% - descrição de um protocolo para avaliação do desempenho das técnicas
% implementadas.
%
%

\chapter{Metodologia de desenvolvimento}\label{sec:meto}

\section{Introdução}

Em um ambiente de virtualização gerenciado pela ferramenta \emph{libVirt}
(apresentada na seção \ref{sec:libvirt}), o software executará em plano de
fundo (ou seja, será um \emph{daemon}) e estará periodicamente coletando
informações quantitativas a respeito do tempo de uso de processador de
cada uma das máquinas virtuais que fazem parte de um determinado
\emph{domínio}. O software utilizará estas informações para treinar um
mecanismo de aprendizado de máquina. Utilizando este mecanismo, a
ferramenta poderá então prever em quais situações alguma máquina virtual
exigirá mais uso de processador que a máquina \emph{host} em que ela
executa terá condições de atender sem prejudicar as outras.

\section{Modelagem do software}

\section{Treinamento}


\section{Avaliação de desempenho}\label{sec:desemp}

% utilizar SVM, K-NN, AR(16) (se for fácil de implementar)
% variar o tamanho das janelas de histórico
% utilizar uma instância do classificador para todas as VMs, versus
% utilizar uma instância nova para cada VM


\chapter{Considerações Finais}

Até o momento foi possível identificar a validade das hipóteses
apresentadas, conhecer em detalhes as técnicas necessárias suas áreas de
estudo e aplicabilidade, e modelar um plano de trabalho para a
implementação do projeto proposto.

Espera-se que este trabalho possa concluir seu objetivo e validar as
hipóteses. Além disso, espera-se que a implementação ajude a conhecer
melhor as características da implementação destas técnicas na prática.


%
% Cronograma
%

\chapter{Cronograma}

Este é o cronograma de desenvolvimento proposto:

\begin{samepage}
\begin{tabular}{|c|c|c|c|c|c|}
\hline
FASES DO ESTUDO & Fev & Mar & Abr & Mai & Jun \\
\hline
Implementação do projeto & X & X & X & & \\
\hline
Testes &   &   &   & X & X \\
\hline
\textbf{Entrega final} &   &   &   &   & X \\
\hline
\end{tabular}
\end{samepage}


%
% Glossário
%
\chapter{Glossário}

\begin{description}
\item[KVM] Kernel-based Virtual Machine.
\item[MV] Máquina Virtual.
\item[SVM] do ingles \emph{Support Vector Machine} ou Máquinas de Vetores
de Suporte, apresentados na seção \ref{sec:svm}.
\item[VM] do inglês \emph{Virtual Machine}, Máquina Virtual.
\end{description}


\bibliography{biblio}

\end{document}
